\documentclass{amsart}
\usepackage{amssymb}
\usepackage{enumerate}
\usepackage[headheight=12pt,textwidth=7in,top=1in, bottom=1in]{geometry}

%some custom commands you may find useful

\usepackage{xparse}
\DeclareDocumentCommand{\diff}{O{} m}{
    \frac{\mathrm{d} #1}{\mathrm{d}#2}
    }
\DeclareDocumentCommand{\pdiff}{O{} m}{
    \frac{\partial #1}{\partial #2}
    }
\DeclareDocumentCommand{\integral}{O{} O{} m O{x}}{
    \int_{#1}^{#2} #3\ \mathrm{d}#4
    }

\def\name{Connor Boyle} %your name goes here
\def\CM{2239} %your cm goes here
\def\hwnum{3} %homework number goes here

%these packages create the footer and page numbering

\usepackage{fancyhdr}
\usepackage{lastpage}
\pagestyle{fancy}
\lhead{\name}
\chead{MA 366: Homework Set \hwnum}
\rhead{\CM}
\fancyfoot[C]{\footnotesize Page \thepage\ of \pageref{LastPage}}

\fancypagestyle{firststyle}
{  \renewcommand{\headrulewidth}{0pt}%
   \fancyhf{}%
   \fancyfoot[C]{\footnotesize Page \thepage\ of \pageref{LastPage}}
}

\begin{document}
\noindent
\thispagestyle{firststyle}
\begin{tabular}{l}
{\LARGE \textbf{MA 366: Real Analysis} }\\
{\Large Homework Set \hwnum}
\end{tabular} \hfill \begin{tabular}{r}
                        \name \\
                        \CM
                        \end{tabular}

\noindent \hrulefill \\\\
On this Homework Set, I worked with Lauren Wiseman.\\
\begin{enumerate}[1.]
\item Let $a, b \in \mathbb R$ with $a<b$.
    \begin{enumerate}[(a)]
    \item Prove that $(a,b)\sim(0,1)$.
    \item Prove that $(a,b)\sim(0,\infty)$.\\\\
    \end{enumerate} 

(a) Prove that $(a,b) \sim (0,1)$
\begin{proof}
Note: Two sets are similar if there exists a bijection between them. Consider the function $f: (a,b) \rightarrow (0,1)$
\[
    f(x) = \frac{x-a}{b-a}
\]
We now need to show that f is bijective. \\
Claim: f is injective.\\
Assume $f(x) = f(y)$ where $x,y\in(a,b)$. Then
\begin{align*}
    f(x) &= f(y) \\
    \frac{x-a}{b-a} &= \frac{y-a}{b-a} \\
    x - a &= y - a \\
    x &= y
\end{align*}
Thus, $f$ is injective. \\
Claim: $f$ is surjective. \\
Pick any arbitrary value $z \in (0,1)$. Then consider $f(w) = z$ for some $w\in(a,b)$:
\begin{align*}
    \frac{w - a}{b - a} &= z \\
    w - a &= z\cdot (b - a) \\
    w &= z \cdot (b - a) + a
\end{align*}
So $f(z\cdot (b-a) + a) = z$ for all $z \in (0,1)$.
Thus, $f$ is surjective. \\
Since $f$ is both injective and surjective, $f$ is bijective. 
Therefore, $(a,b) \sim (0,1)$.
\end{proof}

(b) Prove that $(a,b) \sim (0,\infty)$
\begin{proof}
Note: we know that $(a,b) \sim (0,\infty)$. If we can show that
$(0,1) \sim (0,\infty)$, then by transitivity: $(a,b)\sim (0,\infty)$.
So we aim to show that $(0,1) \sim (0,\infty)$. \\\\
Consider the function $f: (0,1) \rightarrow (0, \infty)$:
\[
    f(x) = \frac{1}{x} - 1    
\]
Claim: $f$ is injective. \\
Assume $f(x) = f(y)$ where $x,y\in(0,1)$. Then
\begin{align*}
    f(x) &= f(y) \\
    \frac{1}{x} - 1 &= \frac{1}{y} - 1 \\
    \frac{1}{x} &= \frac{1}{y} \\
    x &= y
\end{align*}
Thus $f$ is injective. \\
Claim: $f$ is surjective. \\
Pick any arbitrary $z\in(0,\infty)$. Then consider
$f(w) = z$ for some $w\in(0,1)$:
\begin{align*}
    \frac{1}{w} - 1 &= z \\
    \frac{1}{w} &= z + 1 \\
    1 &= w\cdot (z + 1) \\
    \frac{1}{z+1} &= w
\end{align*}
So $f(\frac{1}{z+1}) = z$ for all $z\in(0,\infty)$. 
Thus, $f$ is surjective.\\
Since $f$ is both injective and surjective, $f$ is bijective. Hence
$(0,1) \sim (0,\infty)$. \\
Since $(a,b) \sim (0,1)$ and $(0,1) \sim (0,\infty)$, by transitivity
$(a,b) \sim (0,\infty)$.
\end{proof}
\newpage

\item Consider the set $X=\mathbb R$, then for $a,b\in \mathbb R$,
\begin{enumerate}[(a)]
\item $m_1(a,b) =(a-b)^2$ is a metric. \\
\begin{proof}
We aim to show that $m_1(a,b) > 0$ for $a \neq b$ and $m_1(a,b) > 0$ for $a = b$. \\
Let $c = a - b \neq 0$. Then $m_1(a,b) = c^2$ where $c \in \mathbb{R}$.
By Rudin Proposition 1.18, $c^2 > 0$ since $c \neq 0$. \\
Now consider the case where $a = b$. Then 
\begin{align*}
    m_1(a,b) &= (a - b)^2 \\
    &= (a - a)^2 \\
    &= 0^2 \\
    &= 0\cdot 0 \\
    &= 0.
\end{align*}
Now, we aim to show that $m_1(a,b) = m_1(b,a)$. \\
\begin{align*}
    m_1(a,b) &= (a - b)^2 \\
    &= a^2 -2\cdot a \cdot b + b^2 \\
    &= b^2 -2 \cdot b \cdot a + a^2 \\
    &= (b - a)^2 \\
    &= m_1(b,a).
\end{align*}
Finally, we aim to show that $m_1(a,b) \leq m_1(a,c) + m_1(c,b)$.
\begin{align*}
    m_1(a,c) + m_1(c,b) &= (a - c)^2 + (c - b)^2 \\
    &= a^2 -2\cdot a\cdot c + c^2 + c^2 -2\cdot c\cdot b + b^2 \\
    &= a^2 + (-2\cdot a \cdot c + 2\cdot c^2 - 2\cdot c\cdot b) + b^2 \\
    &= a^2 - 2\cdot( a \cdot c - c^2 + c\cdot b) + b^2 \\
    &= a^2 - 2\cdot a \cdot( c - \frac{c^2}{a} + \frac{c\cdot b}{a}) + b^2 \\
    &\geq a^2 - 2\cdot a \cdot b + b^2 \\
    &= (a - b)^2 \\
    &= m_1(a,b).
\end{align*}
Thus, $m_1$ is a metric.
\end{proof}
\end{enumerate}
\newpage


\end{enumerate}
\end{document} 