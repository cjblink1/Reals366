\documentclass{amsart}
\usepackage{amssymb}
\usepackage{enumerate}
\usepackage[headheight=12pt,textwidth=7in,top=1in, bottom=1in]{geometry}

%some custom commands you may find useful

\usepackage{xparse}
\DeclareDocumentCommand{\diff}{O{} m}{
    \frac{\mathrm{d} #1}{\mathrm{d}#2}
    }
\DeclareDocumentCommand{\pdiff}{O{} m}{
    \frac{\partial #1}{\partial #2}
    }
\DeclareDocumentCommand{\integral}{O{} O{} m O{x}}{
    \int_{#1}^{#2} #3\ \mathrm{d}#4
    }

\def\name{Connor Boyle} %your name goes here
\def\CM{2239} %your cm goes here
\def\hwnum{2} %homework number goes here

%these packages create the footer and page numbering

\usepackage{fancyhdr}
\usepackage{lastpage}
\pagestyle{fancy}
\lhead{\name}
\chead{MA 366: Homework Set \hwnum}
\rhead{\CM}
\fancyfoot[C]{\footnotesize Page \thepage\ of \pageref{LastPage}}

\fancypagestyle{firststyle}
{  \renewcommand{\headrulewidth}{0pt}%
   \fancyhf{}%
   \fancyfoot[C]{\footnotesize Page \thepage\ of \pageref{LastPage}}
}

\begin{document}
\noindent
\thispagestyle{firststyle}
\begin{tabular}{l}
{\LARGE \textbf{MA 366: Real Analysis} }\\
{\Large Homework Set \hwnum}
\end{tabular} \hfill \begin{tabular}{r}
                        \name \\
                        \CM
                        \end{tabular}

\noindent \hrulefill \\\\
On this Homework Set, I worked with Lauren Wiseman.\\
\begin{enumerate}[1.]
\item \emph{Let $x,r \in \mathbb{R}$ with $r \in \mathbb{Q}$, $x \not \in \mathbb{Q}$, and $r \neq 0$.
\begin{enumerate}[(a)]
\item Prove that $rx \not \in \mathbb{Q}$.
\item If $y \in \mathbb{R} \setminus \mathbb{Q}$, determine if $xy$ \emph{must} be irrational. Prove your claim. \\
\end{enumerate}}

(a) Prove that $rx\notin\mathbb{Q}$.
\begin{proof}
Note: $\mathbb{Q}$ is closed under multiplication and multiplicative inverse. \\
Assume, for contradiction, that $rx = p \in\mathbb{Q}$. Then:
\begin{align*}
    rx &= p \\
    \left(\frac{1}{r}\right)\cdot rx &= \left(\frac{1}{r}\right)\cdot p \\
    \left(\frac{r}{r}\right)\cdot x &= \left(\frac{p}{r}\right) \\
    1\cdot x &= \frac{p}{r}\\
    x &= \frac{p}{r}
\end{align*}
Since $p\in\mathbb{Q}$ and $r\in\mathbb{Q}$, then $\frac{p}{r}\in\mathbb{Q}$. 
Thus, $x\in\mathbb{Q}$. But this is absurd! Therefore $rx\notin\mathbb{Q}$.
\end{proof}

(b) The statement "If $y \in \mathbb{R} \setminus \mathbb{Q}$, $xy$ \emph{must} be irrational" is false. 
\begin{proof}
Note: $\mathbb{Q}$ is closed under multiplicative inverse. \\
Additionally, 
\begin{align*}
    \frac{1}{\sqrt{2}} &\text{ is irrational. Since } \\
    \frac{1}{\sqrt{2}} &= \frac{1}{\sqrt{2}}\cdot 1 \\
    &= \frac{1}{\sqrt{2}}\cdot \left(\frac{\sqrt{2}}{\sqrt{2}}\right) \\
    &= \frac{1\cdot\sqrt{2}}{\sqrt{2}\cdot\sqrt{2}} \\
    &= \frac{\sqrt{2}}{2} \text{ which is irrational by (a).}
\end{align*}
Consider:
\begin{align*}
    x = \sqrt{2} &\text{ and } y = \frac{1}{\sqrt{2}} \\
    \text{So } xy &= \sqrt{2}\cdot\left(\frac{1}{\sqrt{2}}\right) \\
    &= \frac{\sqrt{2}}{\sqrt{2}} \\
    &= 1 \text{ (which is rational).}
\end{align*}
In this case, $x$ and $y$ are both irrational, But $xy$ is rational.
Therefore $xy$ does not have to be irrational.
\end{proof}
\newpage

\item \emph{Let $X$ be a non-empty subset of an ordered set $S$. Suppose that $\gamma \in S$ is a lower bound of $X$ and $\psi \in S$ is an upper bound of $X$.
\begin{enumerate}[(a)]
\item Prove that $\gamma \leq \psi$.
\item Determine if the statement $\gamma < \psi$ is \emph{always} true. Prove your claim.\\
\end{enumerate}}
(a) Prove that $\gamma \leq \psi$.
\begin{proof}
Note: $S$ is on ordered set, so $S$ has the transitivity property. \\
Since $\gamma$ is a lower bound of $X$, then $\forall x \in X$, 
$\gamma\leq x$. \\
Similarly, since $\psi$ is an upper bound of $X$, then $\forall x \in X$,
$x \leq\psi$. \\
By transitivity of S:
\begin{align*}
    \forall x\in X, \gamma \leq x \leq \psi \\
    \text {so } \gamma \leq \psi.
\end{align*}

\end{proof}
(b) The statement "$\gamma<\psi$ is \emph{always} true" is false.
\begin{proof}
Note: $S$ is on ordered set, so $S$ has the trichotomy property. \\
Consider, for example, $a\in S$ and $X = \{a\}$. Note that $X$ is non-empty.
Additionally, consider that $\gamma = a$ and $\psi = a$.
By definition of lower bound, $\gamma$ is a lower bound of X.
And by definition of upper bound, $\psi$ is an upper bound of X.
We can see by transitivity of equality:
\begin{align*}
    \gamma &= a = \psi \\
    \text{so } \gamma &= \psi
\end{align*}
By the trichotomy property, $ \gamma \nless \psi$.
\end{proof}

\newpage

\item \emph{Let $B$ be a nonempty set of real numbers that is bounded below. Define the set 
\[ -B = \{ -x : x \in B \}.\]
Prove that $\inf(B) = -\sup(-B)$.}
\begin{proof}
Note: Since B is a nonempty set of real numbers which is bounded below,
then $\inf(B)$ exists.

Let $\inf(B) = \gamma\in\mathbb{R}$. We aim to show that $-\gamma = \sup(-B)$.
First, let us show that $-\gamma$ is an upper bound for $-B$.
Since $\gamma$ is a lower bound of $B$, then:
\begin{align*}
    \forall x\in B \text{, } \gamma &\leq x \\
    (-1)\cdot\gamma &\geq (-1)\cdot x \text{ (by Proposition 1.18c in Rudin)} \\
    -\gamma \geq -x.
\end{align*} 
Thus, by the definition of $-B$ and the definition of upper bound,
$-\gamma$ is an upper bound for $B$. \\

It remains to be shown that $-\gamma$ is the \emph{least} upper bound of $-B$.
This means that for every $\epsilon > 0$, there should exist an $\alpha$ such that
\[
    -\gamma -\epsilon < \alpha < -\gamma \text{ and }\alpha \in -B
\]
If we can find such an $\alpha$ for every $\epsilon>0$,
then we will have shown that $\gamma$ is the least upper bound 
of $-B$.

Note: $\gamma$ is the greatest lower bound of $B$. Hence, for every
$\epsilon > 0$, there exists a $\delta$ such that:
\[
    \gamma < \delta < \gamma + \epsilon \text{ and } \delta\in B.
\]
Then:
\begin{alignat*}{2}
    \gamma &< &\delta &< \gamma + \epsilon \\
    (-1)\cdot\gamma &> &(-1)\cdot\delta &> (-1)\cdot(\gamma + \epsilon) \text{ (by Proposition 1.18c in Rudin)}\\
    -\gamma &> &-\delta &> -\gamma -\epsilon \\
    -\gamma -\epsilon &< &-\delta &< -\gamma
\end{alignat*}
Since $-\delta\in -B$, choose $\alpha = -\delta$. Thus we have shown that
$-\gamma$ is the \emph{least} upper bound of $-B$.
\end{proof}

\newpage

\end{enumerate}
\end{document} 