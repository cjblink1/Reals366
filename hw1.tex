\documentclass{amsart}
\usepackage{amssymb}
\usepackage{enumerate}
\usepackage[headheight=12pt,textwidth=7in,top=1in, bottom=1in]{geometry}

%some custom commands you may find useful

\usepackage{xparse}
\DeclareDocumentCommand{\diff}{O{} m}{
    \frac{\mathrm{d} #1}{\mathrm{d}#2}
    }
\DeclareDocumentCommand{\pdiff}{O{} m}{
    \frac{\partial #1}{\partial #2}
    }
\DeclareDocumentCommand{\integral}{O{} O{} m O{x}}{
    \int_{#1}^{#2} #3\ \mathrm{d}#4
    }

\def\name{Connor Boyle} %your name goes here
\def\CM{2239} %your cm goes here
\def\hwnum{1} %homework number goes here

%these packages create the footer and page numbering

\usepackage{fancyhdr}
\usepackage{lastpage}
\pagestyle{fancy}
\lhead{\name}
\chead{MA 366: Homework Set \hwnum}
\rhead{\CM}
\fancyfoot[C]{\footnotesize Page \thepage\ of \pageref{LastPage}}

\fancypagestyle{firststyle}
{  \renewcommand{\headrulewidth}{0pt}%
   \fancyhf{}%
   \fancyfoot[C]{\footnotesize Page \thepage\ of \pageref{LastPage}}
}

\begin{document}
\noindent
\thispagestyle{firststyle}
\begin{tabular}{l}
{\LARGE \textbf{MA 366: Real Analysis} }\\
{\Large Homework Set \hwnum}
\end{tabular} \hfill \begin{tabular}{r}
                        \name \\
                        \CM
                        \end{tabular}

\noindent \hrulefill

\begin{enumerate}[1.]
\item \emph{Prove that for any integer $m, m$ is even if and only if $m^2$ is even.}
\begin{proof} .\\
%----------------------------------------------------------- 

First we must show that if $m$ is even then $m^2$ is even.
Assume $m$ is even. Then by definition of even, 
\[\exists n\in\mathbb{N} \text{ such that } m = 2n, \]
Now consider $m^2$:
\begin{align*} 
    m^2 &= (2n)^2 \\
    &= 4n^2
\end{align*}
Let $2n^2 = q$. Since $n\in\mathbb{N}$, then $q\in\mathbb{N}$. 
Then $m^2 = 2q, q\in\mathbb{N}$. Therefore, $m^2$ is even. \\
%-----------------------------------------------------------

Second we must show that if $m^2$ is even then $m$ is even.
Let's show that the contrapositive is true.
So we must show that if $m$ is odd, then $m^2$ is odd.
Assume $m$ is odd. Then by definition of odd, 
\[\exists r\in\mathbb{N} \text{ such that } m = 2r - 1, \]
Now consider $m^2$:
\begin{align*}
    m^2 &= (2r - 1)^2 \\
    &= 4r^2 - 4r + 1 \\
    &= 4r^2 - 4r + 2 - 2 + 1 \\
    &= 2(2r^2 -2r + 1) - 1
\end{align*}
Let $2r^2 - 2r + 1 = p$. Since $r\in\mathbb{N}$, then $p\in\mathbb{N}$.
Then $m^2 = 2p - 1, p\in\mathbb{N}$. Therefore, $m^2$ is odd. \\
%-----------------------------------------------------------
\end{proof}

\newpage

\item \emph{For any real numbers $p$ and $q$, if $p$ is rational and 
$q$ is irrational, then $p + q$ is irrational.}
\begin{proof}
First, we know that $\mathbb{Q}$ is closed under addition. We also know
that $\mathbb{Q}$ is closed under additive inverse. From these axioms, we
can see that $\mathbb{Q}$ is closed under differences. \\

For contradiction, assume $p$ is rational, $q$ is irrational, 
$p + q = r$, and $r$ is rational. We see that
\begin{align*}
    p + q &= r \\
    (-p) + p + q &= (-p) + r \\
    q &= (-p) + r \\
    q &= r - p
\end{align*}
$r - p$ must be rational, since $\mathbb{Q}$ is closed under differences.
Therefore $q$ must be rational. But this contradicts our assumption that
$q$ is irrational! Therefore $p + q$ must be irrational.
\end{proof}


\newpage

\item \emph{The problem statement of problem 3 goes here.}
\begin{proof}
Your proof goes here.
\end{proof}


\newpage

\item \emph{Prove that the set of positive rational numbers $\mathbb{Q}^+$ does not have a smallest element.}
\begin{proof} .\\

First let's define an order on $\mathbb{Q}^+$:
For $q_1, q_2\in\mathbb{Q}^+$ and $q_1=\frac{w}{x}$, 
$q_2=\frac{y}{z}$. Define $q_1 < q_2$ to mean $w\cdot z < x\cdot y$. \\

Now, take any positive rational number $\frac{b}{c}$ 
where $b,c\in\mathbb{Z}^+$. Now consider $\frac{b}{d}$ where $d=c+1$. 
Since $c\in\mathbb{Z}^+$, then $d\in\mathbb{Z}^+$. And since 
$b,d\in\mathbb{Z}^+$ then $\frac{b}{d}\in\mathbb{Q}^+$. \\

We know that $c < c+1$. Since $b > 0$ 
\begin{align*}
    b\cdot c &< b\cdot(c+1) \\
    &< b\cdot d
\end{align*}
Therefore $\frac{b}{d}<\frac{b}{c}$. \\

By this method, if we are given
a positive rational number $\frac{b}{c}$ we can find another positive
rational number $\frac{b}{d}$ such that $\frac{b}{d}<\frac{b}{c}$.
Therefore $\mathbb{Q}^+$ does not have a smallest element.

\end{proof}

\newpage

\item \emph{The problem statement of problem 5 goes here.}
\begin{proof}
Your proof goes here.
\end{proof}

\newpage


\item .\\
(i)\emph{ $\exists x\in\mathbb{R}$ such that $\forall z\in\mathbb{R}$, $x < z$ is false.}
\begin{proof}
Assume that such an $x$ exists. Now consider a $z\in\mathbb{R}$
such that $z = x$. In this case, $x \nless z$ because $ x \nless x$. Therefore there is no such $x$.
\end{proof}
.\\
(ii)\emph{ $\forall x\in\mathbb{R}$, $\exists y,z \in\mathbb{R}$ such that $y < z$ and $y+z = x$ is true.}
\begin{proof}
Consider an arbitrary $x\in\mathbb{R}$. Then let
\begin{align*}
    y &= \frac{x-1}{2}  \text{ and}\\
    z &= \frac{x+1}{2}
\end{align*}
Since we have determined a particular set of values for 
$y$ and $z$ such that $y < z$ and $y+z = x$, we have shown 
the theorem to be true.
\end{proof}

\newpage

\item \emph{If $A\subseteq B$ and $B\subseteq C$, then $A\subseteq C$}
\begin{proof}
Assume $A\subseteq B$ and $B\subseteq C$. Then,
\begin{align*}
    &\text{(1) }\forall x\in A, x\in B \text{ and }\\
    &\text{(2) }\forall y\in B, y\in C
\end{align*}
So choose any $z\in A$. Then $z\in B$ by (1). Since $z\in B$, then
$z\in C$ by (2). By universal generalization, $\forall z\in A$,
$z\in C$. Therefore $A\subseteq C$.
\end{proof}

\newpage

\item \emph{Determine what's wrong with the following "function" 
$f:\mathbb{Q}\to \mathbb{Q}$ denoted by $f(\frac{a}{b})=\frac{1}{b}$.} \\\\
For $\frac{a}{b}, \frac{b}{c}\in \mathbb{Q}$, define:
\[
    \frac{a}{b} = \frac{c}{d} \text{ to mean } a\cdot d = b\cdot d.
\] 
If $f$ is a function, it must send every element in its domain to
\underbar{exactly one} element in its codomain. In other words, if
$x=y$, then $f(x)=f(y)$.\\\\

Now consider
\[
    m := \frac{1}{2} \text{ and } n := \frac{2}{4}    
\]
$m, n\in\mathbb{Q}$ and since $1\cdot 4 = 2\cdot 2$, $m=n$. According
to the definition of $f$,
\[
    f\left(\frac{1}{2}\right) = \frac{1}{2} \text{, and }
    f\left(\frac{2}{4}\right) = \frac{1}{4}.
\]
So $m = n$, but $f(m)\neq f(n)$. Therefore $f$ is not a function.
\end{enumerate}
\end{document} 