\documentclass{amsart}
\usepackage{enumerate}
\usepackage[headheight=12pt,textwidth=7in,top=1in, bottom=1in]{geometry}

%some custom commands you may find useful

\usepackage{xparse}
\DeclareDocumentCommand{\diff}{O{} m}{
    \frac{\mathrm{d} #1}{\mathrm{d}#2}
    }
\DeclareDocumentCommand{\pdiff}{O{} m}{
    \frac{\partial #1}{\partial #2}
    }
\DeclareDocumentCommand{\integral}{O{} O{} m O{x}}{
    \int_{#1}^{#2} #3\ \mathrm{d}#4
    }

\def\name{Connor Boyle} %your name goes here
\def\CM{2239} %your cm goes here
\def\hwnum{1} %homework number goes here

%these packages create the footer and page numbering

\usepackage{fancyhdr}
\usepackage{lastpage}
\pagestyle{fancy}
\lhead{\name}
\chead{MA 366: Homework Set \hwnum}
\rhead{\CM}
\fancyfoot[C]{\footnotesize Page \thepage\ of \pageref{LastPage}}

\fancypagestyle{firststyle}
{  \renewcommand{\headrulewidth}{0pt}%
   \fancyhf{}%
   \fancyfoot[C]{\footnotesize Page \thepage\ of \pageref{LastPage}}
}

\begin{document}
\noindent
\thispagestyle{firststyle}
\begin{tabular}{l}
{\LARGE \textbf{MA 366: Real Analysis} }\\
{\Large Homework Set \hwnum}
\end{tabular} \hfill \begin{tabular}{r}
                        \name \\
                        \CM
                        \end{tabular}

\noindent \hrulefill

\begin{enumerate}[1.]
\item \emph{Prove that for any integer $m, m$ is even if and only if $m^2$ is even.}
\begin{proof} .\\
%----------------------------------------------------------- 

First we must show that if $m$ is even then $m^2$ is even.
Assume $m$ is even. Then by definition of even, 
\[\exists n\in\mathbb{N} \text{ such that } m = 2n, \]
Now consider $m^2$:
\begin{align*} 
    m^2 &= (2n)^2 \\
    &= 4n^2
\end{align*}
Let $2n^2 = q$. Since $n\in\mathbb{N}$, then $q\in\mathbb{N}$. 
Then $m^2 = 2q, q\in\mathbb{N}$. Therefore, $m^2$ is even. \\
%-----------------------------------------------------------

Second we must show that if $m^2$ is even then $m$ is even.
Let's show that the contrapositive is true.
So we must show that if $m$ is odd, then $m^2$ is odd.
Assume $m$ is odd. Then by definition of odd, 
\[\exists r\in\mathbb{N} \text{ such that } m = 2r - 1, \]
Now consider $m^2$:
\begin{align*}
    m^2 &= (2r - 1)^2 \\
    &= 4r^2 - 4r + 1 \\
    &= 4r^2 - 4r + 2 - 2 + 1 \\
    &= 2(2r^2 -2r + 1) - 1
\end{align*}
Let $2r^2 - 2r + 1 = p$. Since $r\in\mathbb{N}$, then $p\in\mathbb{N}$.
Then $m^2 = 2p - 1, p\in\mathbb{N}$. Therefore, $m^2$ is odd. \\
%-----------------------------------------------------------
\end{proof}

\newpage

\item \emph{The problem statement of problem 2 goes here.}
\begin{proof}
Your proof goes here.
\end{proof}


\newpage

\item \emph{The problem statement of problem 3 goes here.}
\begin{proof}
Your proof goes here.
\end{proof}


\newpage

\item \emph{Prove that the set of positive rational numbers $\mathbb{Q}^+$ does not have a smallest element.}
\begin{proof} .\\

First let's define an order on $\mathbb{Q}^+$:
For $q_1, q_2\in\mathbb{Q}^+$ and $q_1=\frac{w}{x}$, 
$q_2=\frac{y}{z}$. Define $q_1 < q_2$ to mean $w\cdot z < x\cdot y$. \\

Now, take any positive rational number $\frac{b}{c}$ 
where $b,c\in\mathbb{Z}^+$. Now consider $\frac{b}{d}$ where $d=c+1$. 
Since $c\in\mathbb{Z}^+$, then $d\in\mathbb{Z}^+$. And since 
$b,d\in\mathbb{Z}^+$ then $\frac{b}{d}\in\mathbb{Q}^+$. \\

We know that $c < c+1$. Since $b > 0$ 
\begin{align*}
    b\cdot c &< b\cdot(c+1) \\
    &< b\cdot d
\end{align*}
Therefore $\frac{b}{d}<\frac{b}{c}$. \\

By this method, if we are given
a positive rational number $\frac{b}{c}$ we can find another positive
rational number $\frac{b}{d}$ such that $\frac{b}{d}<\frac{b}{c}$.
Therefore $\mathbb{Q}^+$ does not have a smallest element.

\end{proof}

\end{enumerate}
\end{document} 