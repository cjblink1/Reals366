\documentclass{amsart}
\usepackage{amssymb}
\usepackage{enumerate}
\usepackage[headheight=12pt,textwidth=7in,top=1in, bottom=1in]{geometry}

%some custom commands you may find useful

\usepackage{xparse}
\DeclareDocumentCommand{\diff}{O{} m}{
    \frac{\mathrm{d} #1}{\mathrm{d}#2}
    }
\DeclareDocumentCommand{\pdiff}{O{} m}{
    \frac{\partial #1}{\partial #2}
    }
\DeclareDocumentCommand{\integral}{O{} O{} m O{x}}{
    \int_{#1}^{#2} #3\ \mathrm{d}#4
    }

\def\name{Connor Boyle} %your name goes here
\def\CM{2239} %your cm goes here
\def\hwnum{6} %homework number goes here

%these packages create the footer and page numbering

\usepackage{fancyhdr}
\usepackage{lastpage}

\newtheorem{theorem}{Theorem}
\newtheorem{lemma}[theorem]{Lemma}

\pagestyle{fancy}
\lhead{\name}
\chead{MA 366: Homework Set \hwnum}
\rhead{\CM}
\fancyfoot[C]{\footnotesize Page \thepage\ of \pageref{LastPage}}

\fancypagestyle{firststyle}
{  \renewcommand{\headrulewidth}{0pt}%
   \fancyhf{}%
   \fancyfoot[C]{\footnotesize Page \thepage\ of \pageref{LastPage}}
}

\begin{document}
\noindent
\thispagestyle{firststyle}
\begin{tabular}{l}
{\LARGE \textbf{MA 366: Real Analysis} }\\
{\Large Homework Set \hwnum}
\end{tabular} \hfill \begin{tabular}{r}
                        \name \\
                        \CM
                        \end{tabular}

\noindent \hrulefill \\\\
On this Homework Set, I worked with Lauren Wiseman and Josh Arroyo.\\
First, here are some useful lemmas:

\newpage
\begin{enumerate}[1.]
\item Determine (with proof) whether the following series converge or diverge
\begin{enumerate}[(a)]
\item $\displaystyle \sum_{k=1}^\infty \sqrt{k+1}-\sqrt{k}$
\item $\displaystyle \sum_{k=1}^\infty \frac{\sqrt{k+1}-\sqrt{k}}{k}$
\end{enumerate}
%-------------- Problem 1 -------------------
\begin{lemma}
    $\displaystyle \sum_{k=1}^\infty \frac{1}{2k}$ diverges.
    \begin{proof}
    \end{proof}
\end{lemma}
\begin{lemma}
    $\displaystyle \sum_{k=1}^\infty \frac{1}{k^{\frac{3}{2}}}$ converges.
    \begin{proof}
    \end{proof}
\end{lemma}

\begin{theorem}
    $\displaystyle \sum_{k=1}^\infty \sqrt{k+1}-\sqrt{k}$ diverges.
    \begin{proof}
        Note: from Lemma 1, we see that
        \[
            \displaystyle \sum_{k=1}^\infty \frac{1}{2k} \text{ diverges}.    
        \]
        Let $k \geq 2$. Then consider:
        \begin{align*}
            \frac{1}{2k} &= \frac{1}{2\sqrt{k^2}} \\
            &\leq \frac{1}{2\sqrt{k+1}} \\
            &= \frac{1}{\sqrt{k+1} + \sqrt{k+1}} \\
            &\leq \frac{1}{\sqrt{k+1} + \sqrt{k}} \\
            &= \frac{1}{\sqrt{k+1} + \sqrt{k}} \cdot \left( \frac{\sqrt{k+1} - \sqrt{k}}{\sqrt{k+1} - \sqrt{k}} \right) \\
            &= \sqrt{k+1} - \sqrt{k}.
        \end{align*}
        Thus, for all $k \geq 2$, then
        \[
            \frac{1}{2k} \leq \sqrt{k + 1} - \sqrt{k}
        \]
        By the comparison test:
        \[
            \sum_{k=1}^\infty \sqrt{k+1}-\sqrt{k} \text{ must also diverge.}
        \]
    \end{proof}
\end{theorem}
\begin{theorem}
    $\displaystyle \sum_{k=1}^\infty \frac{\sqrt{k+1}-\sqrt{k}}{k}$ converges.
    \begin{proof}
        Note: from Lemma 2, we see that
        \[
            \displaystyle \sum_{k=1}^\infty \frac{1}{k^{\frac{3}{2}}} converges.
        \]
        Let $k \geq 1$. Then consider:
        \begin{align*}
            \frac{1}{k^{\frac{3}{2}}} &= \frac{1}{k\sqrt{k}} \\
            &\geq \frac{1}{k\sqrt{k+1}} \\
            &\geq \frac{1}{k(\sqrt{k+1} + \sqrt{k})} \\
            &= \frac{1}{k} \cdot \frac{1}{\sqrt{k+1} + \sqrt{k}} \\
            &= \frac{1}{k} \cdot \frac{1}{\sqrt{k+1} + \sqrt{k}} \cdot \left( \frac{\sqrt{k+1} - \sqrt{k}}{\sqrt{k+1} - \sqrt{k}} \right) \\
            &= \frac{\sqrt{k+1} - \sqrt{k}}{k}
        \end{align*}
        Thus, for all $k \geq 1$, then
        \[
            \frac{1}{k^{\frac{3}{2}}} \geq \frac{\sqrt{k + 1} - \sqrt{k}}{k}
        \]
        By the comparison test:
        \[
            \displaystyle \sum_{k=1}^\infty \frac{\sqrt{k+1}-\sqrt{k}}{k} \text{ must also converge.}
        \]
    \end{proof}
\end{theorem}

\newpage

\item  Let $(a_k)$ be a sequence of non-negative real numbers. Prove:   
\[ \sum_{k=1}^{\infty} a_k < \infty \quad \Rightarrow \quad \sum_{k=1}^{\infty} \frac{\sqrt{a_k}}{k} < \infty. \]
%-------------- Problem 2 -------------------
\begin{lemma}
    For any nonnegative real numbers $a$ and $b$,
    \[
        \sqrt{a \cdot b} \leq \max\{a,b\}    
    \]
    \begin{proof}

    \end{proof}
\end{lemma}
\begin{lemma}
    If for all $k \geq 1$, $a_k \geq 0$ and $b_k \geq 0$ and $\displaystyle \sum_{k=1}^\infty a_k$ converges and $\displaystyle \sum_{k=1}^\infty b_k$ converges, then
    \[
        \displaystyle \sum_{k=1}^\infty a_k + b_k \text{ converges.}
    \]
    \begin{proof}

    \end{proof}
\end{lemma}
\begin{lemma}
    If for all $k \geq 1$, $a_k \geq 0$ and $b_k \geq 0$ and $\displaystyle \sum_{k=1}^\infty a_k$ converges and $\displaystyle \sum_{k=1}^\infty b_k$ converges, then
    \[
        \displaystyle \sum_{k=1}^\infty \max\{a_k, b_k\} \text{ converges.}
    \]
    \begin{proof}

    \end{proof}
\end{lemma}
\begin{theorem} 
    If $(a_k)$ is a sequence of non-negative real numbers,
    \[ \sum_{k=1}^{\infty} a_k < \infty \quad \Rightarrow \quad \sum_{k=1}^{\infty} \frac{\sqrt{a_k}}{k} < \infty. \]
    \begin{proof}
        Assume $\sum_{k=1}^{\infty} a_k < \infty$. Then we know that $\sum_{k=1}^{\infty} a_k < \infty$ converges.
        We also know that $\sum_{k=1}^{\infty} \frac{1}{k^2}$ converges.

        Consider the sequence 
        \begin{align*}
            \sum_{k=1}^{\infty} \frac{\sqrt{a_k}}{k} &= \sum_{k=1}^{\infty} \sqrt{\frac{1}{k^2} \cdot a_k } \\
            &\leq \sum_{k=1}^{\infty} \max \left\{ \frac{1}{k^2}, a_k \right\}
        \end{align*}
    \end{proof}
\end{theorem}



\newpage
\item Suppose $(a_n)$ and $(b_n)$ are sequences of positive numbers. Let $\displaystyle L= \lim_{n\to\infty}\frac{a_n}{b_n}$. Prove the following:
\begin{enumerate}[(a)]
\item If $0<L<\infty$, then 
\[ \sum_{n=1}^{\infty} a_n < \infty \quad \Leftrightarrow \quad \sum_{n=1}^{\infty} b_n < \infty.\]
\item If $L=0$, then 
\[ \sum_{n=1}^{\infty} b_n < \infty \quad \Rightarrow \quad \sum_{n=1}^{\infty} a_n < \infty.\]
\end{enumerate}
%-------------- Problem 3 -------------------
Part a: $\sum_{n=1}^{\infty} (L - \epsilon)\cdot b_n < a_n < (L + \epsilon)\cdot b_n$

\newpage
\item Suppose $(a_k)$ is a sequence of real numbers. Define the sequence $(m_k)$ where for all $k \in \mathbb{N}$ we have
\[ m_k= \frac{a_1+a_2+\dots+ a_k}{k}.\]
\begin{enumerate}[(a)]
\item If $a_k \to a$, show that $m_k\to a$ as well.
\item Make a sequence $a_k$ where $a_k$ divergent but $m_k$ converges.
\item Show that there exists a sequence with $a_k>0$ for all $k$ with $\limsup a_k=\infty$ and $m_k\to 0.$
\end{enumerate}
%-------------- Problem 4 -------------------
Part b: $a_k = \frac{1}{2} \cdot (1 - (-1)^k)$


\newpage
\item Let $E \subseteq \mathbb{R}$ defined by 
\[ E = (-1,1) \cup (1,2) \cup (3,5) \cup \{6\}.\]
Now, let $f: E \to \mathbb{R}$ be the function defined by 
\[ f(x)=\frac{x^3-2x^2-5x+6}{x^2-4x+3},\]
Determine the following limits or explain why they don't exist. Prove your claims with an $\varepsilon-\delta$ proof. 
\begin{enumerate}[(a)]
\item $\displaystyle \lim_{x\to 1} f(x)$.
\item $\displaystyle \lim_{x\to 6} f(x)$.
\item $\displaystyle \lim_{x\to 4} f(x)$.
\end{enumerate}
%-------------- Problem 5 -------------------

\newpage
\item  Let $(X,d)$ be the metric space where $X$ is any non-empty set and
\[ d(x,y) = \begin{cases} 1 & x\neq y \\ 
                          0 & \text{otherwise.}
                          \end{cases}\]
Determine which functions $f: (X,d) \rightarrow (\mathbb{R},d_{\infty})$ are continuous. 
%-------------- Problem 6 -------------------
\begin{theorem}
    All functions $f: X \to \mathbb{R}$ are closed.
    \begin{proof}
        Note: $f: X \to \mathbb{R}$ is continuous if and only if for every closed set $V \subseteq \mathbb{R}$, $f^{-1}(V) \subseteq X$ is also closed in $X$. 
    \end{proof}
\end{theorem}
\newpage
\item  Let $f:\mathbb{R} \to \mathbb{R}$ and $a \in \mathbb{R}$ such that
\[ \lim_{x \to a} f(x) = \alpha > 0.\]
Show that there exists an $r>0$ such that 
\[ 0 < |x-a|<r \quad \Rightarrow \quad f(x) > 0.\]
%-------------- Problem 7 -------------------
Note: pick $\alpha > \epsilon$

\end{enumerate}
\end{document} 