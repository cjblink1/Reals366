\documentclass{amsart}
\usepackage{amssymb}
\usepackage{enumerate}
\usepackage[headheight=12pt,textwidth=7in,top=1in, bottom=1in]{geometry}

%some custom commands you may find useful

\usepackage{xparse}
\DeclareDocumentCommand{\diff}{O{} m}{
    \frac{\mathrm{d} #1}{\mathrm{d}#2}
    }
\DeclareDocumentCommand{\pdiff}{O{} m}{
    \frac{\partial #1}{\partial #2}
    }
\DeclareDocumentCommand{\integral}{O{} O{} m O{x}}{
    \int_{#1}^{#2} #3\ \mathrm{d}#4
    }

\def\name{Connor Boyle} %your name goes here
\def\CM{2239} %your cm goes here
\def\hwnum{6} %homework number goes here

%these packages create the footer and page numbering

\usepackage{fancyhdr}
\usepackage{lastpage}

\newtheorem{theorem}{Theorem}
\newtheorem{lemma}[theorem]{Lemma}

\pagestyle{fancy}
\lhead{\name}
\chead{MA 366: Homework Set \hwnum}
\rhead{\CM}
\fancyfoot[C]{\footnotesize Page \thepage\ of \pageref{LastPage}}

\fancypagestyle{firststyle}
{  \renewcommand{\headrulewidth}{0pt}%
   \fancyhf{}%
   \fancyfoot[C]{\footnotesize Page \thepage\ of \pageref{LastPage}}
}

\begin{document}
\noindent
\thispagestyle{firststyle}
\begin{tabular}{l}
{\LARGE \textbf{MA 366: Real Analysis} }\\
{\Large Homework Set \hwnum}
\end{tabular} \hfill \begin{tabular}{r}
                        \name \\
                        \CM
                        \end{tabular}

\noindent \hrulefill \\\\
On this Homework Set, I worked with Lauren Wiseman and Josh Arroyo.\\
\begin{enumerate}[1.]
\item Determine (with proof) whether the following series converge or diverge
\begin{enumerate}[(a)]
\item $\displaystyle \sum_{k=1}^\infty \sqrt{k+1}-\sqrt{k}$
\item $\displaystyle \sum_{k=1}^\infty \frac{\sqrt{k+1}-\sqrt{k}}{k}$
\end{enumerate}
%-------------- Problem 1 -------------------
\begin{lemma}
    $\displaystyle \sum_{k=1}^\infty \frac{1}{2k}$ diverges.
    \begin{proof}
        Note:
        \[
            \sum_{k = 1}^\infty \frac{1}{2k} = \sum_{k = 1}^\infty \frac{1}{2} \cdot \frac{1}{k} = \frac{1}{2} \sum_{k = 1}^\infty \frac{1}{k}   
        \]
        By Rudin Thm. 3.28, we see that $\displaystyle \sum_{k = 1}^\infty \frac{1}{k} = \infty$. Thus,
        $\displaystyle \frac{1}{2} \sum_{k = 1}^\infty \frac{1}{k} = \frac{1}{2} \cdot \infty = \infty$.
        Therefore $\displaystyle \sum_{k=1}^\infty \frac{1}{2k}$ diverges.
    \end{proof}
\end{lemma}
\begin{lemma}
    $\displaystyle \sum_{k=1}^\infty \frac{1}{k^{\frac{3}{2}}}$ converges.
    \begin{proof}
        This follows from Rudin Thm. 3.28.
    \end{proof}
\end{lemma}

\begin{theorem}
    $\displaystyle \sum_{k=1}^\infty \sqrt{k+1}-\sqrt{k}$ diverges.
    \begin{proof}
        Note: from Lemma 1, we see that
        \[
            \displaystyle \sum_{k=1}^\infty \frac{1}{2k} \text{ diverges}.    
        \]
        Let $k \geq 2$. Then consider:
        \begin{align*}
            \frac{1}{2k} &= \frac{1}{2\sqrt{k^2}} \\
            &\leq \frac{1}{2\sqrt{k+1}} \\
            &= \frac{1}{\sqrt{k+1} + \sqrt{k+1}} \\
            &\leq \frac{1}{\sqrt{k+1} + \sqrt{k}} \\
            &= \frac{1}{\sqrt{k+1} + \sqrt{k}} \cdot \left( \frac{\sqrt{k+1} - \sqrt{k}}{\sqrt{k+1} - \sqrt{k}} \right) \\
            &= \sqrt{k+1} - \sqrt{k}.
        \end{align*}
        Thus, for all $k \geq 2$, then
        \[
            \frac{1}{2k} \leq \sqrt{k + 1} - \sqrt{k}
        \]
        By the comparison test:
        \[
            \sum_{k=1}^\infty \sqrt{k+1}-\sqrt{k} \text{ must also diverge.}
        \]
    \end{proof}
\end{theorem}
\begin{theorem}
    $\displaystyle \sum_{k=1}^\infty \frac{\sqrt{k+1}-\sqrt{k}}{k}$ converges.
    \begin{proof}
        Note: from Lemma 2, we see that
        \[
            \displaystyle \sum_{k=1}^\infty \frac{1}{k^{\frac{3}{2}}} converges.
        \]
        Let $k \geq 1$. Then consider:
        \begin{align*}
            \frac{1}{k^{\frac{3}{2}}} &= \frac{1}{k\sqrt{k}} \\
            &\geq \frac{1}{k\sqrt{k+1}} \\
            &\geq \frac{1}{k(\sqrt{k+1} + \sqrt{k})} \\
            &= \frac{1}{k} \cdot \frac{1}{\sqrt{k+1} + \sqrt{k}} \\
            &= \frac{1}{k} \cdot \frac{1}{\sqrt{k+1} + \sqrt{k}} \cdot \left( \frac{\sqrt{k+1} - \sqrt{k}}{\sqrt{k+1} - \sqrt{k}} \right) \\
            &= \frac{\sqrt{k+1} - \sqrt{k}}{k}
        \end{align*}
        Thus, for all $k \geq 1$, then
        \[
            \frac{1}{k^{\frac{3}{2}}} \geq \frac{\sqrt{k + 1} - \sqrt{k}}{k}
        \]
        By the comparison test:
        \[
            \displaystyle \sum_{k=1}^\infty \frac{\sqrt{k+1}-\sqrt{k}}{k} \text{ must also converge.}
        \]
    \end{proof}
\end{theorem}

\newpage

\item  Let $(a_k)$ be a sequence of non-negative real numbers. Prove:   
\[ \sum_{k=1}^{\infty} a_k < \infty \quad \Rightarrow \quad \sum_{k=1}^{\infty} \frac{\sqrt{a_k}}{k} < \infty. \]
%-------------- Problem 2 -------------------
\begin{lemma}
    For nonnegative real numbers $a$ and $b$, if  $a \leq b$, then $\sqrt{a} \leq \sqrt{b}$.
    \begin{proof}
        (by contrapositive)
        We aim to show that if $\sqrt{a} > \sqrt{b}$, then $a > b$.
        Assume $\sqrt{a} > \sqrt{b}$. Then consider
        \begin{align*}
            a &= (\sqrt{a})^2 \\
            &= (\sqrt{a})\cdot (\sqrt{a}) \\
            &> (\sqrt{a})\cdot (\sqrt{b}) \\
            &> (\sqrt{b})\cdot (\sqrt{b}) \\
            &= (\sqrt{b})^2 \\
            &= b
        \end{align*}
        Thus, if $a \leq b$, then $\sqrt{a} \leq \sqrt{b}$.
    \end{proof}
\end{lemma}
\begin{lemma}
    For any nonnegative real numbers $a$ and $b$,
    \[
        \sqrt{a \cdot b} \leq \max\{a,b\}    
    \]
    \begin{proof}
        Consider the following cases:
        \begin{enumerate}
            \item $a$ and $b$ are nonnegative real numbers with $a \leq b$.
            Then 
            \begin{align*}
                a &\leq b \\
                \sqrt{a} &\leq \sqrt{b} \\
                \sqrt{a} \cdot \sqrt{b} &\leq \sqrt{b} \cdot \sqrt{b} \\
                \sqrt{a \cdot b} &\leq \sqrt{b \cdot b} \\
                \sqrt{a \cdot b} &\leq b
            \end{align*}
            \item $a$ and $b$ are nonnegative real numbers with $b \leq a$.
            Since $a$ and $b$ are arbitrary, then the same argument from part (a) applies here.
            So $\sqrt{a \cdot b} \leq a$. \\
        \end{enumerate}
        Thus, in either case, $\sqrt{a \cdot b} \leq \max\{a,b\}$.
    \end{proof}
\end{lemma}
\begin{lemma}
    If for all $k \geq 1$, $a_k \geq 0$ and $b_k \geq 0$ and $\displaystyle \sum_{k=1}^\infty a_k$ converges and $\displaystyle \sum_{k=1}^\infty b_k$ converges, then
    \[
        \displaystyle \sum_{k=1}^\infty a_k + b_k \text{ converges.}
    \]
    \begin{proof}
        Consider the sequences of partial sums $(A_n)$ and $(B_n)$ where:
        \[
            (A_n) = \displaystyle \sum_{k=1}^n a_k \text{ and } (B_n) = \displaystyle \sum_{k=1}^n b_k
        \]
        Since the two series converge, we know that 
        \begin{align*}
            \lim_{n \to \infty} A_n \text{ exists} \\
            \lim_{n \to \infty} B_n \text{ exists}
        \end{align*}
        From limit laws, we know that
        \[
            \lim_{n \to \infty} A_n +  \lim_{n \to \infty} B_n  =  \lim_{n \to \infty} A_n + B_n 
        \]
        Thus, 
        \[
            \lim_{n \to \infty} A_n + B_n \text{ exists.}    
        \]
        So 
        \begin{align*}
            \lim_{n \to \infty} \displaystyle \sum_{k=1}^n a_k &+ \displaystyle \sum_{k=1}^n b_k \text{ exists.} \\
            \lim_{n \to \infty} \displaystyle \sum_{k=1}^n a_k &+ a_k \text{ exists.} \\
            \displaystyle \sum_{k=1}^{\infty} a_k &+ a_k \text{ converges.}
        \end{align*}
    \end{proof}
\end{lemma}
\begin{lemma}
    If for all $k \geq 1$, $a_k \geq 0$ and $b_k \geq 0$ and $\displaystyle \sum_{k=1}^\infty a_k$ converges and $\displaystyle \sum_{k=1}^\infty b_k$ converges, then
    \[
        \displaystyle \sum_{k=1}^\infty \max\{a_k, b_k\} \text{ converges.}
    \]
    \begin{proof}
        By Lemma 7, we know that
        \[
            \displaystyle \sum_{k=1}^{\infty} a_k + a_k \text{ converges.}
        \]
        We also know that for all $k$
        \[
            \max\{a_k, b_k\} \leq a_k + b_k.
        \]
        Thus, by the comparison test, 
        \[
            \displaystyle \sum_{k=1}^\infty \max\{a_k, b_k\} \text{ must also converge.}
        \]
    \end{proof}
\end{lemma}
\begin{theorem} 
    If $(a_k)$ is a sequence of non-negative real numbers,
    \[ \sum_{k=1}^{\infty} a_k < \infty \quad \Rightarrow \quad \sum_{k=1}^{\infty} \frac{\sqrt{a_k}}{k} < \infty. \]
    \begin{proof}
        Assume $\sum_{k=1}^{\infty} a_k < \infty$. Then we know that $\sum_{k=1}^{\infty} a_k $ converges.
        We also know that $\sum_{k=1}^{\infty} \frac{1}{k^2}$ converges.

        Consider the sequence 
        \begin{align*}
            \sum_{k=1}^{\infty} \frac{\sqrt{a_k}}{k} &= \sum_{k=1}^{\infty} \sqrt{\frac{1}{k^2} \cdot a_k } \\
            &\leq \sum_{k=1}^{\infty} \max \left\{ \frac{1}{k^2}, a_k \right\} \text{ by Lemma 6.} \\
        \end{align*}
        By Lemma 8, we know that 
        \[
            \sum_{k=1}^{\infty} \max \left\{ \frac{1}{k^2}, a_k \right\} \text{ converges.}
        \]
        Thus, by the comparison test,
        \[
            \sum_{k=1}^{\infty} \frac{\sqrt{a_k}}{k} \text{ must also converge.}
        \]
        So 
        \[
            \sum_{k=1}^{\infty} \frac{\sqrt{a_k}}{k} < \infty
        \]
    \end{proof}
\end{theorem}



\newpage
\item Suppose $(a_n)$ and $(b_n)$ are sequences of positive numbers. Let $\displaystyle L= \lim_{n\to\infty}\frac{a_n}{b_n}$. Prove the following:
\begin{enumerate}[(a)]
\item If $0<L<\infty$, then 
\[ \sum_{n=1}^{\infty} a_n < \infty \quad \Leftrightarrow \quad \sum_{n=1}^{\infty} b_n < \infty.\]
\item If $L=0$, then 
\[ \sum_{n=1}^{\infty} b_n < \infty \quad \Rightarrow \quad \sum_{n=1}^{\infty} a_n < \infty.\]
\end{enumerate}
%-------------- Problem 3 -------------------
\begin{lemma}
    Let $c \in \mathbb{R^+}$ be an arbitrary constant. Then
    \[
        \sum_{n=1}^{\infty} a_n \text{ converges }\Rightarrow \sum_{n=1}^{\infty} a_n\cdot c \text{ converges }   
    \]
    \begin{proof}
        Suppose that $\sum_{n=1}^{\infty} a_n$ converges to some value $\alpha$. Consider the sequence of partial sums: 
        \[
            A_n = \sum_{k=1}^{n} a_k
        \]
        Then we know that $(A_n)$ converges to $\alpha$. So for any $\epsilon > 0$, we know that there exists an $N \in \mathbb{N}$
        such that if $n \geq N$
        \[
            |A_n - \alpha| < \epsilon.    
        \]
        Let $c \in \mathbb{R^+}$ be an arbitrary constant. Consider the sequence 
        \[
            B_n = \sum_{k=1}^n a_n\cdot c    
        \]
        We aim to show that $(B_n)$ converges. 

        By the distributive law,
        \[
            B_n = \sum_{k=1}^n a_n\cdot c = c\cdot\sum_{k=1}^n a_n = c\cdot A_n
        \]
        By Rudin Thm. 3.3b, we see that 
        \[
            \lim_{n\to\infty} B_n = \lim_{n\to\infty} c\cdot A_n = c\cdot\alpha    
        \]
        Therefore, $(B_n)$ converges.
    \end{proof}
\end{lemma}
\begin{lemma}
    If $\displaystyle \lim_{n \to \infty} \frac{a_n}{b_n} = L$, then for all $\epsilon > 0$, there exists an $N\in\mathbb{N}$
    such that if $n \geq N$
    \[
        b_n \cdot(L - \epsilon) < a_n < b_n\cdot(L + \epsilon). 
    \]
    \begin{proof}
        Assume $\displaystyle \lim_{n \to \infty} \frac{a_n}{b_n} = L$. Then for all $\epsilon > 0$, there exists an $N \in \mathbb{N}$
        such that if $n \geq N$, then
        \[
            \left| \frac{a_n}{b_n} - L \right| < \epsilon    
        \]
        This implies
        \begin{align*}
            -\epsilon &< \frac{a_n}{b_n} - L < \epsilon \\
            L - \epsilon &< \frac{a_n}{b_n} < L + \epsilon \\
            b_n \cdot (L - \epsilon) &< a_n < b_n \cdot (L + \epsilon) \\
        \end{align*}
    \end{proof}
\end{lemma}
\begin{theorem}
    Suppose $(a_n)$ and $(b_n)$ are sequences of positive numbers. Let $\displaystyle L= \lim_{n\to\infty}\frac{a_n}{b_n}$ and $0 < L < \infty$.
    Then 
    \[ \sum_{n=1}^{\infty} b_n < \infty \quad \Rightarrow \quad \sum_{n=1}^{\infty} a_n < \infty.\]
    \begin{proof}
        Assume $\displaystyle \sum_{n=1}^{\infty} b_n < \infty$. Then $\displaystyle \sum_{n=1}^{\infty} b_n$ converges.
        Since $\displaystyle \lim_{n \to \infty} \frac{a_n}{b_n} = L$, we see from Lemma 11 that for all $\epsilon > 0$, there
        exists an $N\in\mathbb{N}$ such that if $n \geq N$,
        \[
            a_n < b_n \cdot (L + \epsilon).  
        \]
        Since $\displaystyle \sum_{n=1}^{\infty} b_n$ converges, we know from Lemma 10 that
        \[
            \displaystyle \sum_{n=1}^{\infty} b_n \cdot (L + \epsilon) \text{ converges.}
        \]
        Therefore, by the comparison test,
        \[
            \displaystyle \sum_{n=1}^{\infty} a_n \text{ converges.}
        \]
    \end{proof}
\end{theorem}
\begin{theorem}
    Suppose $(a_n)$ and $(b_n)$ are sequences of positive numbers. Let $\displaystyle L= \lim_{n\to\infty}\frac{a_n}{b_n}$ and $0 < L < \infty$.
    Then 
    \[ \sum_{n=1}^{\infty} a_n < \infty \quad \Rightarrow \quad \sum_{n=1}^{\infty} b_n < \infty.\]
    \begin{proof}
        Assume $\displaystyle \sum_{n = 1}^\infty a_n < \infty$. Then $\displaystyle \sum_{n = 1}^\infty a_n$ converges.
        Since $\displaystyle \lim_{n \to \infty} \frac{a_n}{b_n} = L$, we see from Lemma 11 that for all $\epsilon > 0$, there
        exists an $N\in\mathbb{N}$ such that if $n \geq N$,
        \[
            b_n \cdot (L - \epsilon) < a_n.
        \]
        Since $\displaystyle \sum_{n = 1}^\infty a_n$ converges, by the comparison test,
        \[
            \sum_{n = 1}^\infty b_n \cdot (L - \epsilon) \text{ converges.}
        \]
        By Lemma 10, $\displaystyle \sum_{n = 1}^\infty b_n$ converges.
    \end{proof}
\end{theorem}
\begin{theorem}
    Suppose $(a_n)$ and $(b_n)$ are sequences of positive numbers. Let $\displaystyle L= \lim_{n\to\infty}\frac{a_n}{b_n}$ and $0 < L < \infty$.
    Then 
    \[ \sum_{n=1}^{\infty} a_n < \infty \quad \Leftrightarrow \quad \sum_{n=1}^{\infty} b_n < \infty.\]
    \begin{proof}
        This claim follows from Theorem 12 and Theorem 13.
    \end{proof}
\end{theorem}
\begin{theorem}
    Suppose $(a_n)$ and $(b_n)$ are sequences of positive numbers. Let $\displaystyle L= \lim_{n\to\infty}\frac{a_n}{b_n}$ and $L = 0$.
    Then 
    \[ \sum_{n=1}^{\infty} b_n < \infty \quad \Rightarrow \quad \sum_{n=1}^{\infty} a_n < \infty.\]
    \begin{proof}
        Assume $\displaystyle \sum_{n = 1}^\infty b_n < \infty$. Then $\displaystyle \sum_{n = 1}^\infty$ converges.
        Since $\displaystyle \lim_{n \to \infty} \frac{a_n}{b_n} = L$, we see from Lemma 11 that for all $\epsilon > 0$, there
        exists an $N\in\mathbb{N}$ such that if $n \geq N$,
        \[
            a_n < b_n \cdot (L + \epsilon).
        \]
        Since $L = 0$, we see that
        \[
            a_n < b_n \cdot \epsilon    
        \]
        
    \end{proof}
\end{theorem}
\newpage
\item Suppose $(a_k)$ is a sequence of real numbers. Define the sequence $(m_k)$ where for all $k \in \mathbb{N}$ we have
\[ m_k= \frac{a_1+a_2+\dots+ a_k}{k}.\]
\begin{enumerate}[(a)]
\item If $a_k \to a$, show that $m_k\to a$ as well.
\item Make a sequence $a_k$ where $a_k$ divergent but $m_k$ converges.
\item Show that there exists a sequence with $a_k>0$ for all $k$ with $\limsup a_k=\infty$ and $m_k\to 0.$
\end{enumerate}
%-------------- Problem 4 -------------------
Part b: $a_k = \frac{1}{2} \cdot (1 - (-1)^k)$


\newpage
\item Let $E \subseteq \mathbb{R}$ defined by 
\[ E = (-1,1) \cup (1,2) \cup (3,5) \cup \{6\}.\]
Now, let $f: E \to \mathbb{R}$ be the function defined by 
\[ f(x)=\frac{x^3-2x^2-5x+6}{x^2-4x+3},\]
Determine the following limits or explain why they don't exist. Prove your claims with an $\varepsilon-\delta$ proof. 
\begin{enumerate}[(a)]
\item $\displaystyle \lim_{x\to 1} f(x)$.
\item $\displaystyle \lim_{x\to 6} f(x)$.
\item $\displaystyle \lim_{x\to 4} f(x)$.
\end{enumerate}
%-------------- Problem 5 -------------------
\begin{lemma}
    $a = 1$ is a limit point of $E$.
    \begin{proof}
        Since $(-1, 1)$ is an open interval of $\mathbb{R}$, $a = 1$ is a limit point of $(-1, 1)$. 
        Thus, for all $\epsilon > 0$, there exists an $x \in (-1, 1)$ such that
        \[
            x \in N_\epsilon (a) \text{ and } x \neq a.   
        \] 
        But $(-1, 1)$ is contained in $E$. So $x \in E$. 
        So for all $\epsilon >0$, there exists an $x \in E$ such that 
        \[
            x \in N_\epsilon (a) \text{ and } x \neq a.
        \] 
        Therefore, $a = 1$ is a limit point of $E$.
    \end{proof}
\end{lemma}
\begin{theorem}
    $\displaystyle \lim_{x \to 1} f(x) = 3$.
    \begin{proof}
        Note: from Lemma 16 we see that $1$ is a limit point of $E$. \\
        Fix $\epsilon > 0$, and choose $\delta > 0$ such that $\delta < \min\{\epsilon, 1\}$.
        Suppose 
        \[
            0 < |x - 1| < \delta    
        \]
        Then we see that 
        \begin{align*}
            |(x + 2) - 3| &< \delta \\
            \left| \frac{(x - 1)\cdot(x + 2)}{(x - 1)} - 3\right| &< \delta
        \end{align*}
        Note: $x - 1 \neq 0$ because $|x - 1| > 0$.
        We also see that 
        \[
            \left| \frac{(x - 1)\cdot(x + 2)\cdot(x - 3)}{(x - 1)\cdot(x - 3)} - 3\right| < \delta
        \]
        Note: $x - 3 \neq 0$ because 
        \begin{align*}
            |1 - 3| &\leq |x - 1| + |x - 3| \\
            2 &< 1 + |x - 3| \\
            |x - 3| &> 1
        \end{align*}
        Finally we see that 
        \begin{align*}
            \left| \frac{x^3-2x^2-5x+6}{x^2-4x+3} - 3 \right| &< \delta \\
            | f(x) - 3 | &< \delta < \epsilon.
        \end{align*}
        Thus, $\displaystyle \lim_{x \to 1} f(x) = 3$.
    \end{proof}
\end{theorem}
\begin{lemma}
    $a = 6$ is not a limit point of $E$.
    \begin{proof}
        Consider $epsilon = \frac{1}{2}$. By the definition of $E$ there exists no $x \in E$ such that
        \[
            x \in N_\epsilon(a) \text{ and } x \neq a    
        \]
        Thus, $a = 6$ is not a limit point of $E$.
    \end{proof}
\end{lemma}
\newpage
\begin{theorem}
    $\displaystyle \lim_{x \to 6} f(x)$ does not exist.
    \begin{proof}
        In order for $\displaystyle \lim_{x \to 6} f(x)$ to exist, $a = 6$ must be a limit point
        of $E$. By Lemma 18, we see that $a = 6$ is not a limit point of $E$. Therefore 
        $\displaystyle \lim_{x \to 6} f(x)$ does not exist.
    \end{proof}
\end{theorem}
\begin{lemma}
    $a = 4$ is a limit point of $E$.
    \begin{proof}
        Fix an $\epsilon$ between $0$ and $1$. Consider the neighborhood $N_\epsilon(a)$. We can always
        find an element
        \[
            x = a + \frac{\epsilon}{2} \in N_\epsilon(a) \subseteq (3, 5).
        \]
        But $(3, 5) \subseteq E$. So for all $\epsilon > 0$, there exists an $x \in E$ such that 
        \[
            x \in N_\epsilon(a) \text{ and } x \neq a.    
        \]
        Therefore, $a = 4$ is a limit point of $E$.
    \end{proof}
\end{lemma}
\begin{theorem}
    $\displaystyle \lim_{x \to 4} f(x) = 6$.
    \begin{proof}
        Note: from Lemma 20 we see that $4$ is a limit point of $E$. \\
        Fix $\epsilon > 0$, and choose $\delta > 0$ such that $\delta < \min\{\epsilon, 1\}$.
        Suppose 
        \[
            0 < |x - 4| < \delta    
        \]
        Then we see that 
        \begin{align*}
            |(x + 2) - 6| &< \delta \\
            \left| \frac{(x - 1)\cdot(x + 2)}{(x - 1)} - 6\right| &< \delta
        \end{align*}
        Note: $x - 1 \neq 0$ because 
        \begin{align*}
            |1 - 4| &\leq |x - 1| + |x - 4| \\
            3 &< |x - 1| + 1 \\
            |x - 1| &> 2
        \end{align*}
        Note: $x - 3 \neq 0$ because 
        \begin{align*}
            |3 - 4| &\leq |x - 3| + |x - 4| \\
            1 &< |x - 3| + 1 \\
            |x - 3| &> 0
        \end{align*}
        Finally we see that 
        \begin{align*}
            \left| \frac{x^3-2x^2-5x+6}{x^2-4x+3} - 6 \right| &< \delta \\
            | f(x) - 6 | &< \delta < \epsilon.
        \end{align*}
        Thus, $\displaystyle \lim_{x \to 1} f(x) = 6$.
    \end{proof}
\end{theorem}
\newpage
\item  Let $(X,d)$ be the metric space where $X$ is any non-empty set and
\[ d(x,y) = \begin{cases} 1 & x\neq y \\ 
                          0 & \text{otherwise.}
                          \end{cases}\]
Determine which functions $f: (X,d) \rightarrow (\mathbb{R},d_{\infty})$ are continuous. 
%-------------- Problem 6 -------------------
\begin{theorem}
    All functions $f: X \to \mathbb{R}$ are continuous.
    \begin{proof}
        Note: $f: X \to \mathbb{R}$ is continuous if and only if for every closed set $V \subseteq \mathbb{R}$, $f^{-1}(V) \subseteq X$ is also closed in $X$.
        We know from Homework 4 that every subset of $(X,d)$ is closed.
        Consider a closed subset $V \subseteq \mathbb{R}$ and an arbitrary function $f: X \to \mathbb{R}$. The preimage of $V$, 
        $f^{-1} (V)$ is, by definition, contained in $X$. Since every subset of $X$ is closed, then
        $f^{-1} (V)$ must also be closed. Since $V$ was chosen arbitrarily, for every closed set $V \subseteq \mathbb{R}$,  
        $f^{-1} (V)$ is also closed in $X$. Thus, $f$ is continuous. Since $f$ was chosen arbitrarily,
        all functions $f: X \to \mathbb{R}$ are continuous.
    \end{proof}
\end{theorem}
\newpage
\item  Let $f:\mathbb{R} \to \mathbb{R}$ and $a \in \mathbb{R}$ such that
\[ \lim_{x \to a} f(x) = \alpha > 0.\]
Show that there exists an $r>0$ such that 
\[ 0 < |x-a|<r \quad \Rightarrow \quad f(x) > 0.\]
%-------------- Problem 7 -------------------
\begin{proof}
    Suppose $\displaystyle \lim_{x \to a} f(x) = \alpha$. Then we know that for all $\epsilon > 0$, 
    there exists a $\delta > 0$ such that
    \[
        0 < |x - a| < \delta \Rightarrow |f(x) - \alpha| < \epsilon.    
    \]
    Since $\alpha > 0$, consider any $\epsilon'$ such that $0 < \epsilon' < \alpha$. Then we know there must exist some $\delta' > 0$ such that
    \[
        0 < |x - a| < \delta' \Rightarrow |f(x) - \alpha| < \epsilon' < \alpha.    
    \]
    Thus, let $r = \delta'$. Then if $0 < |x - a| < r$, 
    \begin{align*}
        |f(x) - \alpha| &< \alpha \\    
        f(x) - \alpha &> -\alpha \\
        f(x) > 0.
    \end{align*}
\end{proof}
\end{enumerate}
\end{document} 