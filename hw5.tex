\documentclass{amsart}
\usepackage{amssymb}
\usepackage{enumerate}
\usepackage[headheight=12pt,textwidth=7in,top=1in, bottom=1in]{geometry}

%some custom commands you may find useful

\usepackage{xparse}
\DeclareDocumentCommand{\diff}{O{} m}{
    \frac{\mathrm{d} #1}{\mathrm{d}#2}
    }
\DeclareDocumentCommand{\pdiff}{O{} m}{
    \frac{\partial #1}{\partial #2}
    }
\DeclareDocumentCommand{\integral}{O{} O{} m O{x}}{
    \int_{#1}^{#2} #3\ \mathrm{d}#4
    }

\def\name{Connor Boyle} %your name goes here
\def\CM{2239} %your cm goes here
\def\hwnum{5} %homework number goes here

%these packages create the footer and page numbering

\usepackage{fancyhdr}
\usepackage{lastpage}

\newtheorem{theorem}{Theorem}
\newtheorem{lemma}[theorem]{Lemma}

\pagestyle{fancy}
\lhead{\name}
\chead{MA 366: Homework Set \hwnum}
\rhead{\CM}
\fancyfoot[C]{\footnotesize Page \thepage\ of \pageref{LastPage}}

\fancypagestyle{firststyle}
{  \renewcommand{\headrulewidth}{0pt}%
   \fancyhf{}%
   \fancyfoot[C]{\footnotesize Page \thepage\ of \pageref{LastPage}}
}

\begin{document}
\noindent
\thispagestyle{firststyle}
\begin{tabular}{l}
{\LARGE \textbf{MA 366: Real Analysis} }\\
{\Large Homework Set \hwnum}
\end{tabular} \hfill \begin{tabular}{r}
                        \name \\
                        \CM
                        \end{tabular}

\noindent \hrulefill \\\\
On this Homework Set, I worked with Lauren Wiseman.\\
\begin{enumerate}[1.]
\item Give a clean analysis proof (using the $\epsilon,N$-definition of a limit) to show that the following sequence converges in $(\mathbb{R},d_{\infty})$:
\[ x_n = \frac{2n^2+3n-1}{7n^2+8n+3}. \]
%-------------- Problem 1 -------------------
\begin{lemma}
    Any constant sequence $(x_n)$ where $x_n = c$ converges to $c$.
    \begin{proof}
        Fix $\epsilon > 0$. Choose $N = 1$. Then consider $d(x_n, c)$ for all $n \geq N$:
        \begin{align*}
            d(x_n, c) &= d(c, c) \\
            &= 0 \\
            &< \epsilon
        \end{align*}
    \end{proof}
\end{lemma}
Consider the following sequences and the values to which they converge:
\begin{align}
    x_n &= 2 &\text{ converges to } x = 2 \text{ by Lemma 1}\\
    x_n &= 7 + \frac{8}{n} + \frac{3}{n^2} &\text{ converges to } x = 7 \text{ by Lemma 1, Rudin Theorem 3.3a, and Rudin Theorem 3.20a} \\
    x_n &= \frac{3}{n} &\text{ converges to } x = 0 \text{ by Rudin Theorem 3.3b, and Rudin Theorem 3.20a}\\
    x_n &= -\frac{1}{n^2} &\text{ converges to } x = 0 \text{ by Rudin Theorem 3.3b, and Rudin Theorem 3.20a} \\
    x_n &= \frac{1}{7 + \frac{8}{n} + \frac{3}{n^2}} &\text{ converges to } x = \frac{1}{7} \text{ by (2) and Rudin Theorem 3.3d} \\
    x_n &= (2)\cdot \left(\frac{1}{7 + \frac{8}{n} + \frac{3}{n^2}}\right) &\text{ converges to } x = 2\cdot\frac{1}{7} = \frac{2}{7} \text{ by (1), (5), and Rudin Theorem 3.3c} \\
    x_n &= \left(\frac{3}{n}\right)\cdot\left(\frac{1}{7 + \frac{8}{n} + \frac{3}{n^2}} \right) &\text{ converges to } x = 0 \cdot \frac{1}{7} = 0 \text{ by (3), (5), and Rudin Theorem 3.3c} \\
    x_n &= \left(-\frac{1}{n^2}\right)\cdot\left(\frac{1}{7 + \frac{8}{n} + \frac{3}{n^2}} \right) &\text{ converges to } x = 0 \cdot \frac{1}{7} = 0 \text{ by (4), (5), and Rudin Theorem 3.3c}
\end{align}
And by (6), (7), (8), and Rudin Theorem 3.3a:
\[
    (2)\cdot \left(\frac{1}{7 + \frac{8}{n} + \frac{3}{n^2}}\right) +
    \left(\frac{3}{n}\right)\cdot\left(\frac{1}{7 + \frac{8}{n} + \frac{3}{n^2}} \right) +
    \left(-\frac{1}{n^2}\right)\cdot\left(\frac{1}{7 + \frac{8}{n} + \frac{3}{n^2}} \right)
    \text{ converges to } \frac{2}{7} + 0 + 0 = \frac{2}{7}
\]
Upon simplification, we see that:
\begin{align*}
    &(2)\cdot \left(\frac{1}{7 + \frac{8}{n} + \frac{3}{n^2}}\right) +
    \left(\frac{3}{n}\right)\cdot\left(\frac{1}{7 + \frac{8}{n} + \frac{3}{n^2}} \right) +
    \left(-\frac{1}{n^2}\right)\cdot\left(\frac{1}{7 + \frac{8}{n} + \frac{3}{n^2}} \right) \\
    &= (2n^2)\cdot \left(\frac{1}{7n^2 + 8n + 3}\right) +
    \left(3n\right)\cdot\left(\frac{1}{7n^2 + 8n + 3} \right) +
    \left(-1\right)\cdot\left(\frac{1}{7n^2 + 8n + 3} \right) \\
    &= (2n^2 + 3n - 1)\cdot \left(\frac{1}{7n^2 + 8n + 3} \right) \\
    &= \frac{2n^2 + 3n - 1}{7n^2 + 8n + 3}
\end{align*}
Thus:
\[
    x_n = \frac{2n^2 + 3n - 1}{7n^2 + 8n + 3} \text{ converges to } x = \frac{2}{7}
\]
\newpage

\item  If a sequence $(x_n)$ in $\mathbb{R}$ converges, prove that the sequence $(|x_n|)$ also converges. Determine, with proof, if the convergence of $(|x_n|)$ guarantees that the sequence $(x_n)$ converges.
%-------------- Problem 2 -------------------





\newpage
\item Suppose that $(x_n)$ is a Cauchy sequence in a metric space $(X,d)$. Further suppose that a subsequence $(x_{n_k})$ converges to a point $x \in X$. Prove that $x_n \xrightarrow{n \to \infty} x$.
%-------------- Problem 3 -------------------
\newpage
\item Suppose that $(x_n)$ and $(y_n)$ are Cauchy sequences in a metric space $(X,d)$. Let $(r_n)$ be the sequence of real numbers defined by
\[ r_n = d(x_n, y_n). \]
Show that $(r_n)$ converges.
%-------------- Problem 4 -------------------
\newpage
\item Let $(x_n)$ and $(y_n)$ be sequences of real numbers.
\begin{enumerate}[(a)]
\item Prove that 
\[ \limsup_{n \to \infty} (x_n+y_n) \leq \limsup_{n \to \infty} x_n + \limsup_{n \to \infty} y_n, \]
assuming that the right hand side is not in the indeterminant form $\infty-\infty$.
\item Give an example of sequences $(x_n)$ and $(y_n)$ where
\[ \limsup_{n \to \infty} (x_n+y_n) < \limsup_{n \to \infty} x_n + \limsup_{n \to \infty} y_n.\]
\end{enumerate}
%-------------- Problem 5 -------------------
\newpage
\item  Let $X$ be the set of sequences $\mathbf{x} = (x_n)$ where for all $n \in \mathbb{N}$ we have $x_n \in \{0,1\}$. If $\mathbf{x},\mathbf{y} \in X$ such that $\mathbf{x} \neq \mathbf{y}$, then let
\[ v(\mathbf{x},\mathbf{y}) = \min\{ n \in \mathbb{N}: x_n \neq y_n \}.\]
Now, define $d: X \times X \to \mathbb{R}$ by
\[ d\big( \mathbf{x},\mathbf{y} \big) = \begin{cases} 2^{-v( \mathbf{x},\mathbf{y} )} & \mathbf{x} \neq \mathbf{y} \\ 0 & \mathbf{x}=\mathbf{y}.\end{cases}.\]
\begin{enumerate}[(a)]
\item Show that $(X,d)$ is a metric space.
\item Let $k \in \mathbb{N}$, and let $r=2^{-k}$. For a fixed $\mathbf{x} \in X$, characterize those sequences $\mathbf{y} \in N_{r}(\mathbf{x})$.
\item Show that $X$ is compact.
%-------------- Problem 6 -------------------
\end{enumerate}
\end{enumerate}
\end{document} 