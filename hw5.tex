\documentclass{amsart}
\usepackage{amssymb}
\usepackage{enumerate}
\usepackage[headheight=12pt,textwidth=7in,top=1in, bottom=1in]{geometry}

%some custom commands you may find useful

\usepackage{xparse}
\DeclareDocumentCommand{\diff}{O{} m}{
    \frac{\mathrm{d} #1}{\mathrm{d}#2}
    }
\DeclareDocumentCommand{\pdiff}{O{} m}{
    \frac{\partial #1}{\partial #2}
    }
\DeclareDocumentCommand{\integral}{O{} O{} m O{x}}{
    \int_{#1}^{#2} #3\ \mathrm{d}#4
    }

\def\name{Connor Boyle} %your name goes here
\def\CM{2239} %your cm goes here
\def\hwnum{5} %homework number goes here

%these packages create the footer and page numbering

\usepackage{fancyhdr}
\usepackage{lastpage}

\newtheorem{theorem}{Theorem}
\newtheorem{lemma}[theorem]{Lemma}

\pagestyle{fancy}
\lhead{\name}
\chead{MA 366: Homework Set \hwnum}
\rhead{\CM}
\fancyfoot[C]{\footnotesize Page \thepage\ of \pageref{LastPage}}

\fancypagestyle{firststyle}
{  \renewcommand{\headrulewidth}{0pt}%
   \fancyhf{}%
   \fancyfoot[C]{\footnotesize Page \thepage\ of \pageref{LastPage}}
}

\begin{document}
\noindent
\thispagestyle{firststyle}
\begin{tabular}{l}
{\LARGE \textbf{MA 366: Real Analysis} }\\
{\Large Homework Set \hwnum}
\end{tabular} \hfill \begin{tabular}{r}
                        \name \\
                        \CM
                        \end{tabular}

\noindent \hrulefill \\\\
On this Homework Set, I worked with Lauren Wiseman.\\
First, here are some useful lemmas:
%------------ Lemmas -------------------
\begin{lemma}
    Any constant sequence $(x_n)$ where $x_n = c$ converges to $c$.
    \begin{proof}
        Fix $\epsilon > 0$. Choose $N = 1$. Then consider $d(x_n, c)$ for all $n \geq N$:
        \begin{align*}
            d(x_n, c) &= d(c, c) \\
            &= 0 \\
            &< \epsilon
        \end{align*}
    \end{proof}
\end{lemma}
\begin{lemma}
    For all $a,b \in\mathbb{R}$, $|a\cdot b| = |a|\cdot|b|$
    \begin{proof} (By cases) \\

        Case 1: Assume $a$ and $b$ are both nonnegative. Then we know that $a\cdot b \geq 0$. Thus:
        \begin{align*}
            |a \cdot b| &= a \cdot b \text{ by Rudin Def. 1.32 } \\
            &= |a| \cdot |b| \text{ by Rudin Def. 1.32 } 
        \end{align*}

        Case 2: Assume $a$ and $b$ are both negative. Then we know that $a\cdot b \geq 0$. Thus:
        \begin{align*}
            |a \cdot b| &= a \cdot b \text{ by Rudin Def. 1.32 } \\
            &= (1)\cdot(a)\cdot (b) \\
            &= (-1)\cdot(-1)\cdot(a)\cdot (b) \\
            &= (-1)\cdot(a)\cdot(-1)\cdot (b) \\
            &= (-a)\cdot(-b) \\
            &= |a| \cdot |b| \text{ by Rudin Def. 1.32 } 
        \end{align*}

        Case 3: Assume $a \geq 0$ and $b < 0$. Then we know that $a\cdot b \leq 0$. Thus:
        \begin{align*}
            |a \cdot b| &= (-1) \cdot a \cdot b \text{ by Rudin Def. 1.32 } \\
            &= (a)\cdot(-1)\cdot (b) \\
            &= (a)\cdot(-b) \\
            &= |a| \cdot |b| \text{ by Rudin Def. 1.32 } 
        \end{align*}

        Case 4: Assume $a < 0$ and $b \geq 0$. This case is the same as Case 3 where $a$ and $b$ are swapped. Thus:
        \begin{align*}
            |a \cdot b| &= |a| \cdot |b| \text{ by Case 3.}
        \end{align*}
    \end{proof}
\end{lemma}
\begin{lemma}
    For all $a,b\in\mathbb{R}$, $\left| |a| - |b| \right| \leq |a - b|$.
    \begin{proof}
        By triangle inequality we have:
        \begin{align*}
            |a| &= |a - 0| \\
            &\leq |a - b| + |b - 0| \\
            &= |a - b| + |b| 
        \end{align*}
        On rearranging, we see that:
        \begin{align*}
            |a| - |b| &\leq |a - b| \\
            ||a| - |b|| &\leq ||a - b|| \\
            ||a| - |b|| &\leq |a - b| \text{ since } |a - b| \geq 0
            \text{ and by Rudin Def. 1.32 }\\
        \end{align*}
    \end{proof}
\end{lemma}
\begin{lemma} 
    The following sequence $(x_n)$ converges to $x = 1$:
    \[
        x_n = 1 + \frac{1}{n}    
    \]
    \begin{proof}
        Fix an $\epsilon > 0$. Now, by the archimedian principle, choose a natural number $N$ such that:
        \[
            \epsilon\cdot N > 1    
        \]
        Suppose $n$ is a natural number and $n \geq N$. Then:
        \begin{align*}
            \epsilon\cdot N &> 1 \\
            \epsilon &> \frac{1}{N} \\
            &\geq \frac{1}{n} \\
            &= \frac{1}{n} + 1 - 1 \\
            &= \left| \left(\frac{1}{n} + 1\right) - 1 \right| \\
            &= \left| 1 - \left(\frac{1}{n} + 1\right) \right| \\
            &= d(x_n, 1)
        \end{align*}
        So $(x_n) \to 1$.
    \end{proof}
\end{lemma}
\begin{lemma} 
    The following sequence $(x_n)$ converges to $x = 1$:
    \[
        x_n = 1 - \frac{1}{n}    
    \]
    \begin{proof}
        Fix an $\epsilon > 0$. Now, by the archimedian principle, choose a natural number $N$ such that:
        \[
            \epsilon\cdot N > 1    
        \]
        Suppose $n$ is a natural number and $n \geq N$. Then:
        \begin{align*}
            \epsilon\cdot N &> 1 \\
            \epsilon &> \frac{1}{N} \\
            &\geq \frac{1}{n} \\
            &= \left| -\frac{1}{n}\right| \\
            &= \left| -\frac{1}{n} + 1 - 1\right| \\
            &= \left| \left(1 - \frac{1}{n} \right) - 1 \right| \\
            &= d(x_n, 1)
        \end{align*}
        So $(x_n) \to 1$.
    \end{proof}
\end{lemma}
\begin{lemma}
    Claim:
    \[
        \limsup_{n \to \infty} \left((-1)^n \cdot \left( 1 - \frac{2}{n} \right) \right) = 1.
    \]
    \begin{proof}
       Let 
       \[
            x_n = (-1)^n \cdot \left( 1 - \frac{2}{n} \right).
       \]
       Consider the even subsequence: 
       \[
            x_{2n} = 1 - \frac{2}{2n} = 1 - \frac{1}{n}.
       \]
       We know that this subsequence converges to $1$ from Lemma 5.
       Now, pick any $x > 1$ and choose $N = 2$. For all $n \geq N$, we see that
       \begin{align*}
            x - x_n &= x - (-1)^n \cdot \left( 1 - \frac{2}{n} \right) \\
            &\geq x - \left( 1 - \frac{2}{n} \right) \\
            &> 1 - \left( 1 - \frac{2}{n} \right) \\
            &= \frac{2}{n} \\
            &> 0
       \end{align*}
       Thus, $x_n < x$. By Rudin Thm. 3.17,  $\limsup_{n \to \infty} \left(x_n\right) = 1$
    \end{proof}
\end{lemma}
\begin{lemma} 
    The following sequence $(x_n)$ converges to $x = 1$:
    \[
        x_n = 1 - \frac{2}{2n + 1}    
    \]
    \begin{proof}
        Fix an $\epsilon > 0$. Now, by the archimedian principle, choose a natural number $N$ such that:
        \[
            \epsilon \cdot (2N + 1) > 2
        \]
        Suppose $n$ is a natural number and $n \geq N$. Then:
        \begin{align*}
            \epsilon \cdot (2N + 1) &> 2 \\
            \epsilon &> \frac{2}{2N + 1} \\
            &\geq \frac{2}{2n + 1} \\
            &= \left| -\frac{2}{2n + 1} \right| \\
            &= \left| -\frac{2}{2n + 1} + 1 - 1\right| \\
            &= \left|\left(1  -\frac{2}{2n + 1}\right) - 1\right| \\
            &= d(x_n, 1)
        \end{align*}
        So $(x_n) \to 1$.
    \end{proof}
\end{lemma}
\begin{lemma}
    Claim:
    \[
        \limsup_{n \to \infty} \left((-1)^{n+1} \cdot \left( 1 - \frac{2}{n} \right) \right) = 1.
    \]
    \begin{proof}
       Let 
       \[
            x_n = (-1)^{n+1} \cdot \left( 1 - \frac{2}{n} \right).
       \]
       Consider the odd subsequence: 
       \[
            x_{2n + 1} = 1 - \frac{2}{2n + 1} .
       \]
       We know that this subsequence converges to $1$ from Lemma 7.
       Now, pick any $x > 1$ and choose $N = 2$. For all $n \geq N$, we see that
       \begin{align*}
            x - x_n &= x - (-1)^n \cdot \left( 1 - \frac{2}{n} \right) \\
            &\geq x - \left( 1 - \frac{2}{n} \right) \\
            &> 1 - \left( 1 - \frac{2}{n} \right) \\
            &= \frac{2}{n} \\
            &> 0
       \end{align*}
       Thus, $x_n < x$. By Rudin Thm. 3.17,  $\limsup_{n \to \infty} \left(x_n\right) = 1$
    \end{proof}
\end{lemma}
\newpage
\begin{enumerate}[1.]
\item Give a clean analysis proof (using the $\epsilon,N$-definition of a limit) to show that the following sequence converges in $(\mathbb{R},d_{\infty})$:
\[ x_n = \frac{2n^2+3n-1}{7n^2+8n+3}. \]
%-------------- Problem 1 -------------------

Consider the following sequences and the values to which they converge:
\begin{align}
    x_n &= 2 &\text{ converges to } x = 2 \text{ by Lemma 1}\\
    x_n &= 7 + \frac{8}{n} + \frac{3}{n^2} &\text{ converges to } x = 7 \text{ by Lemma 1, Rudin Theorem 3.3a, and Rudin Theorem 3.20a} \\
    x_n &= \frac{3}{n} &\text{ converges to } x = 0 \text{ by Rudin Theorem 3.3b, and Rudin Theorem 3.20a}\\
    x_n &= -\frac{1}{n^2} &\text{ converges to } x = 0 \text{ by Rudin Theorem 3.3b, and Rudin Theorem 3.20a} \\
    x_n &= \frac{1}{7 + \frac{8}{n} + \frac{3}{n^2}} &\text{ converges to } x = \frac{1}{7} \text{ by (2) and Rudin Theorem 3.3d} \\
    x_n &= (2)\cdot \left(\frac{1}{7 + \frac{8}{n} + \frac{3}{n^2}}\right) &\text{ converges to } x = 2\cdot\frac{1}{7} = \frac{2}{7} \text{ by (1), (5), and Rudin Theorem 3.3c} \\
    x_n &= \left(\frac{3}{n}\right)\cdot\left(\frac{1}{7 + \frac{8}{n} + \frac{3}{n^2}} \right) &\text{ converges to } x = 0 \cdot \frac{1}{7} = 0 \text{ by (3), (5), and Rudin Theorem 3.3c} \\
    x_n &= \left(-\frac{1}{n^2}\right)\cdot\left(\frac{1}{7 + \frac{8}{n} + \frac{3}{n^2}} \right) &\text{ converges to } x = 0 \cdot \frac{1}{7} = 0 \text{ by (4), (5), and Rudin Theorem 3.3c}
\end{align}
And by (6), (7), (8), and Rudin Theorem 3.3a:
\[
    (2)\cdot \left(\frac{1}{7 + \frac{8}{n} + \frac{3}{n^2}}\right) +
    \left(\frac{3}{n}\right)\cdot\left(\frac{1}{7 + \frac{8}{n} + \frac{3}{n^2}} \right) +
    \left(-\frac{1}{n^2}\right)\cdot\left(\frac{1}{7 + \frac{8}{n} + \frac{3}{n^2}} \right)
    \text{ converges to } \frac{2}{7} + 0 + 0 = \frac{2}{7}
\]
Upon simplification, we see that:
\begin{align*}
    &(2)\cdot \left(\frac{1}{7 + \frac{8}{n} + \frac{3}{n^2}}\right) +
    \left(\frac{3}{n}\right)\cdot\left(\frac{1}{7 + \frac{8}{n} + \frac{3}{n^2}} \right) +
    \left(-\frac{1}{n^2}\right)\cdot\left(\frac{1}{7 + \frac{8}{n} + \frac{3}{n^2}} \right) \\
    &= (2n^2)\cdot \left(\frac{1}{7n^2 + 8n + 3}\right) +
    \left(3n\right)\cdot\left(\frac{1}{7n^2 + 8n + 3} \right) +
    \left(-1\right)\cdot\left(\frac{1}{7n^2 + 8n + 3} \right) \\
    &= (2n^2 + 3n - 1)\cdot \left(\frac{1}{7n^2 + 8n + 3} \right) \\
    &= \frac{2n^2 + 3n - 1}{7n^2 + 8n + 3}
\end{align*}
Thus:
\[
    x_n = \frac{2n^2 + 3n - 1}{7n^2 + 8n + 3} \text{ converges to } x = \frac{2}{7}
\]
\newpage

\item  If a sequence $(x_n)$ in $\mathbb{R}$ converges, prove that the sequence $(|x_n|)$ also converges. Determine, with proof, if the convergence of $(|x_n|)$ guarantees that the sequence $(x_n)$ converges.
%-------------- Problem 2 -------------------

\begin{theorem} If $(x_n)$ converges, $(|x_n|)$ converges.
    \begin{proof}
        Note: Since $(x_n)$ is a seqence of reals, and $(x_n)$ converges, by Rudin Thm. 3.11a, $(x_n)$ must be Cauchy. Thus,
        for all $\epsilon > 0$, there exists a natural number $N$ such that 
        \[
            \text{ for all } m,n \geq N, \text{ } d(x_n, x_m) < \epsilon.
        \]
        Since we are working with the usual metric, for all $m,n \geq N$:
        \begin{align*}
            |x_n - x_m| &< \epsilon \\
            ||x_n| - |x_m|| &\leq \text{ by Lemma 3}. \\
            d(|x_n|,|x_m|) &\leq
        \end{align*}
        So we see that for all $\epsilon > 0$, there exists a natural number $N$ such that
        \[
            \text{ for all } m,n \geq N, \text{ } d(|x_n|, |x_m|) < \epsilon.
        \]
        Therefore, the sequence $(|x_n|)$ is Cauchy. \\
        Since $(|x_n|)$ is a sequence of reals,
        by Rudin Thm. 3.11c, $(|x_n|)$ converges.
    \end{proof}
\end{theorem}

\begin{theorem} The statement "the convergence of $(|x_n|)$ guarantees that the sequence $(x_n)$ converges" is false.
    \begin{proof}
        In order to show that the above statement is false, we aim to show that there exists a sequence
        $(x_n)$ such that $(|x_n|)$ converges, but $(x_n)$ does not converge. \\
        
        Consider the following sequence:
        \[
            x_n = (-1)^n\cdot \left(1 + \frac{1}{n} \right)    
        \]
        From the class notes on Sequences, this sequence diverges. However, consider the sequence $(|x_n|)$ where:
        \begin{align*}
            |x_n| &= \left|(-1)^n\cdot \left(1 + \frac{1}{n} \right)\right| \\
            &= |(-1)^n|\cdot \left|\left(1 + \frac{1}{n} \right)\right| \text{ by Lemma 2 } \\
            &= (1) \cdot \left|\left(1 + \frac{1}{n} \right)\right| \\
            &= \left|\left(1 + \frac{1}{n} \right)\right| \\
            &= 1 + \frac{1}{n}
        \end{align*}
        By Lemma 4, $(|x_n|)$ converges to $x = 1$.

        Thus, we have found a sequence $(x_n)$ such that $(|x_n|)$ converges, but $(x_n)$ does not converge.
    \end{proof}
\end{theorem}

\newpage
\item Suppose that $(x_n)$ is a Cauchy sequence in a metric space $(X,d)$. Further suppose that a subsequence $(x_{n_k})$ converges to a point $x \in X$. Prove that $x_n \xrightarrow{n \to \infty} x$.
%-------------- Problem 3 -------------------
\begin{theorem} If $(x_n)$ is a Cauchy sequence, and a subsequence $(x_{n_k})$ converges to a point $x \in X$, then $x_n \xrightarrow{n \to \infty} x$.
    \begin{proof}
        Note: since $(x_{n_k})$ converges, then for any $\epsilon > 0$, there exists a natural number $N_1$ such that for all
        $n_k \geq N_1$, 
        \[
            d(x_{n_k}, x) < \frac{\epsilon}{2}.
        \]
        Also note: since $(x_n)$ is Cauchy, then for any $\epsilon > 0$, there exists a natural number $N_2$ such that for all
        $n,m \geq N_2$, 
        \[
            d(x_n, x_m) < \frac{\epsilon}{2}.
        \]
        So fix an $\epsilon > 0$. Then let $N = \max \{N_1, N_2\}$. By the triangle inequality, for all $n, n_k \geq N$:
        \begin{align*}
            d(x_n, x) &\leq d(x_n, x_{n_k}) + d(x_{n_k}, x) \\
            &< \frac{\epsilon}{2} + \frac{\epsilon}{2} \\
            &< \epsilon
        \end{align*}
        So by defnition of convergence, $x_n \xrightarrow{n \to \infty} x$. 
    \end{proof}
\end{theorem}
\newpage
\item Suppose that $(x_n)$ and $(y_n)$ are Cauchy sequences in a metric space $(X,d)$. Let $(r_n)$ be the sequence of real numbers defined by
\[ r_n = d(x_n, y_n). \]
Show that $(r_n)$ converges.
%-------------- Problem 4 -------------------
\begin{theorem} Claim: $(r_n)$ converges.
    \begin{proof}
        Note: since $(x_n)$ is Cauchy, then for any $\epsilon > 0$, there exists a natural number $N_x$ such that
        for all $n,m \geq N_x$:
        \[
            d(x_n, x_m) < \frac{\epsilon}{2}
        \]
        Similarly note: since $(y_n)$ is Cauchy, then for any $\epsilon > 0$, there exists a natural number $N_y$ such that
        for all $k,l \geq N_y$:
        \[
            d(y_k, y_l) < \frac{\epsilon}{2}
        \]
        Finally note: $(X,d)$ is a metric space, so let $N = \max \{N_x, N_y\}$, and $p,q \geq N$. Then:
        \begin{align*}
            r_p = d(x_p, y_p) &\leq d(x_p, x_q) + d(x_q, y_q) + d(y_q, y_p) \\
            &< \frac{\epsilon}{2} + r_q + \frac{\epsilon}{2} \\
            &= \epsilon + r_q
        \end{align*}
        Fix an $\epsilon > 0$. Let $N = \max \{N_x, N_y\}$, and choose $p,q \geq N$ such that $r_p \geq r_q$. Then we know that:
        \begin{align*}
            r_p &< \epsilon + r_q \\
            r_p - r_q &< \epsilon \\
            |r_p - r_q| &< \epsilon \\
            d(r_p, r_q) &< \epsilon
            % d(r_p, r_q) &= |r_p - r_q| \\
            % &= |d(x_p, y_p) - d(x_q, y_q)| \\
            % &\leq |d(x_p, x_q) + d(x_q, y_q) + d(y_q, y_p) - d(x_q, y_q)| \\
            % &= |d(x_p, x_q) + d(y_q, y_p) + d(x_q, y_q) - d(x_q, y_q)| \\
            % &= d(x_p, x_q) + d(y_q, y_p) + 0
        \end{align*}
        Thus, $(r_n)$ is Cauchy. Since $(r_n)$ is a sequence of real numbers and $(r_n)$ is Cauchy, 
        by Rudin Thm 3.11x, $(r_n)$ converges.
    \end{proof}
\end{theorem}
\newpage
\item Let $(x_n)$ and $(y_n)$ be sequences of real numbers.
\begin{enumerate}[(a)]
\item Prove that 
\[ \limsup_{n \to \infty} (x_n+y_n) \leq \limsup_{n \to \infty} x_n + \limsup_{n \to \infty} y_n, \]
assuming that the right hand side is not in the indeterminant form $\infty-\infty$.
\item Give an example of sequences $(x_n)$ and $(y_n)$ where
\[ \limsup_{n \to \infty} (x_n+y_n) < \limsup_{n \to \infty} x_n + \limsup_{n \to \infty} y_n.\]
\end{enumerate}
%-------------- Problem 5 -------------------
\begin{theorem}
    \[ \limsup_{n \to \infty} (x_n+y_n) \leq \limsup_{n \to \infty} x_n + \limsup_{n \to \infty} y_n, \]
    assuming that the right hand side is not in the indeterminant form $\infty-\infty$.
    \begin{proof}
        Consider the sequence $(x_n + y_n)$. By Rudin Thm 3.6, $(x_n + y_n)$ contains a
        convergent subsequence, say $(x_{n_k} + y_{n_k})$ which converges to $\limsup_{n \to \infty} (x_n+y_n)$.
        Now, since $(x_{n_k})$ is a sequence of reals, it must also contain a convergent subsequence $(x_{n_{k_a}})$ which converges to some value $x\in\mathbb{R}$.
        Further, since $(y_{n_{k_a}})$ is a sequence of reals, it also has a convergent subsequence $(y_{n_{k_{a_b}}})$ which converges to some value $y \in \mathbb{R}$.
        So let:
        \begin{align*}
            (x_{n_{k_{a_b}}} + y_{n_{k_{a_b}}}) \text{ be the subsequence of } (x_{n} + y_{n}) \text{ which converges to } \limsup_{n \to \infty} (x_n+y_n).
        \end{align*}
        As such, since $x$ and $y$ are subsequential limits of $(x_n)$ and $(y_n)$ respectively,
        \begin{align*}
            \limsup_{n \to \infty} (x_n+y_n) &= x + y \\
            &\leq \limsup_{n \to \infty} x_n + \limsup_{n \to \infty} y_n
        \end{align*}

    \end{proof}
\end{theorem}
For part (b), consider the following sequences $(x_n)$ and $(y_n)$:
\begin{align*}
    x_n &= (-1)^n \cdot \left( 1 - \frac{2}{n} \right) \\
    y_n &= (-1)^{n + 1} \cdot \left( 1 - \frac{2}{n} \right)
\end{align*}
From Lemma 6 and Lemma 8, we know that the $\limsup_{n \to \infty} x_n$ = 1 and the  $\limsup_{n \to \infty} y_n$ = 1. Now consider the sequence
$(x_n + y_n)$:
\begin{align*}
    x_n + y_n &= (-1)^n \cdot \left( 1 - \frac{2}{n} \right) + (-1)^{n + 1} \cdot \left( 1 - \frac{2}{n} \right) \\
    &= (-1)^n \cdot \left( 1 - \frac{2}{n} + (-1) \cdot \left(1 - \frac{2}{n}\right)\right) \\
    &= (-1)^n \cdot 0 \\
    &= 0
\end{align*}
$(x_n + y_n)$ is the constant sequence $0$, which converges to $x + y = 0$, by Lemma 1. Thus, by
Rudin Example 3.18c, $\limsup_{n \to \infty} (x_n + y_n) = 0$.
We see that 
\begin{align*}
    0 &< 1 + 1 \\
    \limsup_{n \to \infty} (x_n + y_n) &<  \limsup_{n \to \infty} x_n + \limsup_{n \to \infty} y_n   
\end{align*}
\newpage
\item  Let $X$ be the set of sequences $\mathbf{x} = (x_n)$ where for all $n \in \mathbb{N}$ we have $x_n \in \{0,1\}$. If $\mathbf{x},\mathbf{y} \in X$ such that $\mathbf{x} \neq \mathbf{y}$, then let
\[ v(\mathbf{x},\mathbf{y}) = \min\{ n \in \mathbb{N}: x_n \neq y_n \}.\]
Now, define $d: X \times X \to \mathbb{R}$ by
\[ d\big( \mathbf{x},\mathbf{y} \big) = \begin{cases} 2^{-v( \mathbf{x},\mathbf{y} )} & \mathbf{x} \neq \mathbf{y} \\ 0 & \mathbf{x}=\mathbf{y}.\end{cases}.\]
\begin{enumerate}[(a)]
\item Show that $(X,d)$ is a metric space.
\item Let $k \in \mathbb{N}$, and let $r=2^{-k}$. For a fixed $\mathbf{x} \in X$, characterize those sequences $\mathbf{y} \in N_{r}(\mathbf{x})$.
\item Show that $X$ is compact.
%-------------- Problem 6 -------------------
\end{enumerate}
\end{enumerate}
\end{document} 