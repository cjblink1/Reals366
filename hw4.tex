\documentclass{amsart}
\usepackage{amssymb}
\usepackage{enumerate}
\usepackage[headheight=12pt,textwidth=7in,top=1in, bottom=1in]{geometry}

%some custom commands you may find useful

\usepackage{xparse}
\DeclareDocumentCommand{\diff}{O{} m}{
    \frac{\mathrm{d} #1}{\mathrm{d}#2}
    }
\DeclareDocumentCommand{\pdiff}{O{} m}{
    \frac{\partial #1}{\partial #2}
    }
\DeclareDocumentCommand{\integral}{O{} O{} m O{x}}{
    \int_{#1}^{#2} #3\ \mathrm{d}#4
    }

\def\name{Connor Boyle} %your name goes here
\def\CM{2239} %your cm goes here
\def\hwnum{4} %homework number goes here

%these packages create the footer and page numbering

\usepackage{fancyhdr}
\usepackage{lastpage}
\pagestyle{fancy}
\lhead{\name}
\chead{MA 366: Homework Set \hwnum}
\rhead{\CM}
\fancyfoot[C]{\footnotesize Page \thepage\ of \pageref{LastPage}}

\fancypagestyle{firststyle}
{  \renewcommand{\headrulewidth}{0pt}%
   \fancyhf{}%
   \fancyfoot[C]{\footnotesize Page \thepage\ of \pageref{LastPage}}
}

\begin{document}
\noindent
\thispagestyle{firststyle}
\begin{tabular}{l}
{\LARGE \textbf{MA 366: Real Analysis} }\\
{\Large Homework Set \hwnum}
\end{tabular} \hfill \begin{tabular}{r}
                        \name \\
                        \CM
                        \end{tabular}

\noindent \hrulefill \\\\
On this Homework Set, I worked with Lauren Wiseman.\\
\begin{enumerate}[1.]
    \item Let $X$ be an infinite set. Define $d: X \times X \to \mathbb{R}$ by
    \[ d(x,y) = \begin{cases} 1 &  x \neq y \\ 
                              0 &  x = y. 
                \end{cases}
    \]
    \begin{enumerate}[(a)]
    \item Prove that $d$ is a metric.

\begin{proof}
Note: if $d$ is a metric, it must obey separation, symmetry, and the
triangle inequality.

First, we aim to show that $d(x,y)$ obeys separation: $d(x,y) > 0$ if
$x \neq y$ and $d(x,y) = 0$ if $x = y$.\\

Case 1: $x \neq y$\\
If $x \neq y$, $d(x,y)$ is defined to be equal to $1$ which is greater than zero.\\

Case 2: $x = y$ \\
If $x = y$, $d(x,y)$ is defined to be equal to zero. \\
Thus, $d(x,y)$ obeys separation. \\

Now, we aim to show that $d(x,y)$ obeys symmetry: $d(x,y) = d(y,x)$. \\

Case 1: $x \neq y$\\
If $x \neq y$, $d(x,y) = d(y,x) = 1$ \\

Case 2: $x = y$\\
If $x = y$, $d(x,y) = d(y,x) = 0$.\\
Thus, $d(x,y)$ obeys symmetry. \\

Finally, we aim to show that $d(x,y)$ obeys the triangle inequality: 
For $a,b,c\in X$, $d(a,b) \leq d(a,c) + d(c,b)$. \\

Suppose, for contradiction that $d(a,b) \nleq d(a,c) + d(c,b)$ for
some values $a,b,c\in X$.
The only way for this to happen, according to the definition of $d$,
is for $d(a,b) = 1$ and $d(a,c) = d(c,b) = 0$. But what would this imply?

\begin{align*}
    d(a,c) = 0 &\rightarrow a = c \\
    d(c,b) = 0 &\rightarrow b = c \\
    d(a,b) = 1 &\rightarrow a \neq b \\
\end{align*}
This is absurd! Therefore $d(x,y)$ obeys the triangle inequality.
Since $d(x,y)$ obeys separation, symmetry, and the triangle inequality,
$d(x,y)$ is a metric.
\end{proof}

        
    \item Determine which subsets of $(X,d)$ are open.\\
Claim: Every subset of $(X, d)$ is open.
\begin{proof}
Note: If a subset $A$ of $(X, d)$ is open, that means that every point in $A$
is an interior point of $A$. That is to say, for each $a\in A$, there exists
an $r > 0$ such that $N_r(a) \subseteq A$. \\\\
First, consider the empty set $\phi \subseteq X$. Since $\phi$ is empty,
it contains no points. Thus, all points of $\phi$ are interior points. \\\\
Second, consider an arbitrary non-empty set $A \subseteq X$. Since
A is non-empty, it must contain some $a \in A$.
Consider $N_\frac{1}{2}(a)$: 
\begin{align*}
    N_\frac{1}{2}(a) &= \{x\in X : d(a,x) < 1/2\}
\end{align*}
But by definition of $d(x,y)$, the only element in $N_\frac{1}{2}(a)$
is $a$ itself. Thus, for all $a\in A$:
\begin{align*}
    N_\frac{1}{2}(a) &= \{a\}
\end{align*}
But since $a\in A$, $N_\frac{1}{2}(a) \subseteq A$. Thus every point of $A$
is an interior point, so A is open. Since $A$ was an arbitrary non-empty subset of $(X,d)$, by universal generalization,
every non-empty subset of $(X,d)$ is open. \\\\
Therefore, every subset of $(X,d)$ is open.
\end{proof}
        
    \item Determine which subsets of $(X,d)$ are closed.

Claim: Every subset of $(X,d)$ is closed.
\begin{proof}
Note: If a subset $A$ of $(X,d)$ is closed, then every limit point of 
$A$ is contained in $A$. \\\\
Lemma: Every subset $A$ of $(X,d)$ has no limit points.
\begin{proof}.\\
Case 1: Consider the empty set $A = \phi$.
Assume, for contradiction, that $A$ has a limit point in $(X,d)$.
Then there exists some point $x\in X$ such that for all $\epsilon > 0$,
$N_\epsilon(x)$ contains an element $a\in A$ and $a \neq x$. But A is the empty set! There
can be no such $a\in A$. Thus, $A$ has no limit points.\\

Case 2: Consider a non-empty subset $A$ of $(X,d)$. Assume, for contradiction,
that $A$ has a limit point in $(X,d)$. Then there exists some point $x \in X$
such that for all $\epsilon > 0$, $N_\epsilon(x)$ contains an element $a \in A$ and 
$a \neq x$. 

Consider $N_\frac{1}{2}(x)$: 
\begin{align*}
    N_\frac{1}{2}(x) &= \{y\in X : d(y,x) < 1/2\}
\end{align*}
By definition of $d(x,y)$, the only element in $N_\frac{1}{2}(x)$
is $x$ itself. Thus:
\begin{align*}
    N_\frac{1}{2}(x) &= \{x\}
\end{align*}
But there is no other point besides $x$ in $N_\frac{1}{2}(x)$! Therefore $x$
cannot be a limit point of $A$. Since $x$ was an arbitrary element in $X$,
by universal generalization, no point in $X$ can be a limit point of $A$. Thus,
$A$ has no limit points. \\

In all cases, A has no limit points. Thus, every subset $A$ of $(X,d)$ has no
limit points.
\end {proof}

Now, consider an arbitrary subset $A$ of $(X,d)$. By the above lemma, A has no limit points.
Consequently, the set the set of limit points of $A$ is the empty set $\phi$.
And since $\phi$ is contained in any set, $\phi \subseteq A$. Thus $A$
contains all of its limit points. Hence $A$ is closed.
Since $A$ was an arbitrary subset of $X$, by universal generalization,
every subset of $(X,d)$ is closed.
\end{proof}
\end{enumerate}
\newpage
\item Let $A$ be the following subset of $(\mathbb{R},d_{\infty})$ (the reals with the usual metric):
\[ A=\left\{ (-1)^n+\frac{1}{m} : n,m \in \mathbb{N} \right\}. \]
Find (and prove your claims!) the limit points and isolated points of the set, and determine whether the set is open, closed or neither.

Claim: The set of $A$'s limit points, $A' = \{-1, 1\}$.
\begin{proof}
We must first show that $1$ and $-1$ are limit points of $A$. \\

Claim: $-1$ is a limit point of $A$. \\
Note: If $-1$ is a limit point of $A$, then for all $\epsilon > 0$,
$N_\epsilon(-1)$ contains a point $a \in A$. \\
Fix $\epsilon > 0$. Now, let
\[
    a = (-1)^n + \frac{1}{m} \text{ for some } n,m \in \mathbb{N}  
\]
In order for $a$ to be in $N_\epsilon(-1)$, then:
\begin{align*}
    d_\infty(-1,a) &< \epsilon \\
    \left| -1 - \left( (-1)^n + \frac{1}{m} \right) \right| &< \epsilon
\end{align*}
So choose $n$ to be any odd integer in $\mathbb{N}$. By the archimedian principle,
choose $m \in \mathbb{N}$ such that $m\cdot \epsilon > 1$. Then:
\begin{align*}
    m\cdot\epsilon &> 1 \\
    \epsilon &> \frac{1}{m} \\
    \epsilon &> \frac{1}{m} + (-1) - (-1) \\
    \epsilon &> \frac{1}{m} + (-1)^n - (-1) \\
    \epsilon &> \left( \frac{1}{m} + (-1)^n \right) - (-1) \\
    \epsilon &> \left| \left( \frac{1}{m} + (-1)^n \right) - (-1) \right| \\
    \epsilon &> \left| (-1) - \left( \frac{1}{m} + (-1)^n \right) \right| \\
    \epsilon &> d_\infty(-1, a)
\end{align*}
By this method, we have shown that we can find an $a \in A$ that 
is also contained in $N_\epsilon(-1)$ for all values of $\epsilon$.
Therefore, $-1$ is a limit point of $A$. \\

Claim: $1$ is a limit point of $A$. \\
Note: If $1$ is a limit point of $A$, then for all $\epsilon > 0$,
$N_\epsilon(1)$ contains a point $a \in A$. \\
Fix $\epsilon > 0$. Now, let
\[
    a = (-1)^n + \frac{1}{m} \text{ for some } n,m \in \mathbb{N}  
\]
In order for $a$ to be in $N_\epsilon(1)$, then:
\begin{align*}
    d_\infty(1,a) &< \epsilon \\
    \left| 1 - \left( (-1)^n + \frac{1}{m} \right) \right| &< \epsilon
\end{align*}
So choose $n$ to be any even integer in $\mathbb{N}$. By the archimedian principle,
choose $m \in \mathbb{N}$ such that $m\cdot \epsilon > 1$. Then:
\begin{align*}
    m\cdot\epsilon &> 1 \\
    \epsilon &> \frac{1}{m} \\
    \epsilon &> \frac{1}{m} + (1) - (1) \\
    \epsilon &> \frac{1}{m} + (-1)^n - (1) \\
    \epsilon &> \left( \frac{1}{m} + (-1)^n \right) - (1) \\
    \epsilon &> \left| \left( \frac{1}{m} + (-1)^n \right) - (1) \right| \\
    \epsilon &> \left| (1) - \left( \frac{1}{m} + (-1)^n \right) \right| \\
    \epsilon &> d_\infty(1, a)
\end{align*}
By this method, we have shown that we can find an $a \in A$ that 
is also contained in $N_\epsilon(1)$ for all values of $\epsilon$.
Therefore, $1$ is a limit point of $A$. \\

Now we must show that there are no other limit points of $A$ besides
$-1$ and $1$.

Claim: If $x \in X$ and $x \neq -1$ and $x \neq 1$, then $x$ is not a limit point of $A$. \\
% Note: If $x$ is a limit point of $A$, then for every $\epsilon > 0$, there must be an 
% infinite number of elements of $A$ in $N_\epsilon(x)$. Consequently, if
% there exists an $\epsilon > 0$ for which there are only a finite number of elements of 
% $A$ in $N_\epsilon(x)$, then $x$ is not a limit point of $A$.

Note: We can write $A = A_1 \cup A_{-1}$ where 
\begin{align*}
    A_1 &= \left\{1+ \frac{1}{m}: m\in \mathbb{N}\right\} \text{ and} \\
    A_{-1} &= \left\{-1+ \frac{1}{m}: m\in \mathbb{N}\right\}
\end{align*}

Assume $x \in X$ and $x \neq -1$ and $x \neq 1$. 
Let $d = \min(|x - 1|, |x + 1|)$ and consider $N_\frac{d}{2}(x)$.
Suppose then that: \\

Case 1: $a \in A_1$ and $a \in N_\frac{d}{2}(x)$.
Then there exists an $m \in \mathbb{N}$ such that
\[
    a = 1 + \frac{1}{m}    
\]
Suppose, for contradiction that $m\cdot \frac{d}{2} > 1$ so that
\[
    \frac{d}{2} > \frac{1}{m} = |a - 1|    
\]
Then
\begin{align*}
    d \leq |x - 1| &\leq |x - a| + |a - 1| \\
    &< \frac{d}{2} + \frac{d}{2}\\
    &= d\\
    d &< d
\end{align*}
This is absurd! Therefore $m\cdot \frac{d}{2} \leq 1$ so that 
\[
    \frac{d}{2} \leq \frac{1}{m}
\]
This means that $A_1 \cap N_\frac{d}{2}(x)$ is finite. \\

Case 2: $a \in A_{-1}$ and $a \in N_\frac{d}{2}(x)$.
Then there exists an $m \in \mathbb{N}$ such that
\[
    a = -1 + \frac{1}{m}    
\]
Suppose, for contradiction that $m\cdot \frac{d}{2} > 1$ so that
\[
    \frac{d}{2} > \frac{1}{m} = |a - (-1)|    
\]
\begin{align*}
    d \leq |x - (-1)| &\leq |x - a| + |a - (-1)| \\
    &< \frac{d}{2} + \frac{d}{2}\\
    &= d\\
    d &< d
\end{align*}
This is absurd! Therefore $m\cdot \frac{d}{2} \leq 1$ so that 
\[
    \frac{d}{2} \leq \frac{1}{m}
\]
This means that $A_{-1} \cap N_\frac{d}{2}(x)$ is finite.

In each case, there can only be a finite number of points in
$A \cap N_\frac{d}{2}(x)$. Thus, $x$ is not a limit point of A.

\end{proof}

Claim: All points in $A$ are isolated points.
\begin{proof}
    Note: We can write $A = A_1 \cup A_{-1}$ where 
    \begin{align*}
        A_1 &= \left\{1+ \frac{1}{m}: m\in \mathbb{N}\right\} \text{ and} \\
        A_{-1} &= \left\{-1+ \frac{1}{m}: m\in \mathbb{N}\right\}
    \end{align*}

Case 1: Consider an arbitrary $a_k \in A_1$ such that
\[
    a_k = 1 + \frac{1}{k}
\]
We aim to show that $a_k$ is not a limit point.
Let 
\begin{align*}
    d_k &= |a_k - a_{k+1}| \\
    &= \left| 1 + \frac{1}{k} - \left(1 + \frac{1}{k+1}\right)\right|\\
    &= \left| \frac{1}{k} - \frac{1}{k+1} \right| \\
    &= \left| \frac{1}{k\cdot (k+1)} \right| \\
    &= \frac{1}{k\cdot (k+1)}
\end{align*}
Consider $N_{d_k}(a_k)$

\end{proof}



All points in A are isolated points.
A is neither open nor closed.


\newpage

\item Let $A \subseteq X$ where $(X,d)$ is some metric space. Define $A'$ to be the set of limit points of $A$.
\begin{enumerate}[(a)]
\item Prove that $A'$ is closed.
\begin{proof} Note: $A'$ is closed if and only if every limit point of $A'$ 
    is contained in $A'$.

    Suppose $A'$ is the empty set. Then $A'$ has no limit points. Thus, every limit point 
    of $A'$ is contained in $A'$.\\

    Now suppose that $A'$ is non-empty and finite. In order for $A'$ to have a limit point
    $x \in X$, any neighborhood around $x$ must contain an infinite number of points of $A'$.
    Since $A'$ is finite, no such point $x$ can exist. Thus, $A'$ has no limit points. Therefore
    every limit point of $A'$ is contained in $A'$. \\
    
    Finally suppose that $A'$ is non-empty and infinite. Let $x$ be a limit point of $ A'$.
    Fix an $\epsilon > 0$. There must be a $y \in A'$ such that $0 < d(x,y) < \frac{\epsilon}{2}$.
    Let $d(x,y) = \omega$. Since $y \in A'$, there must be a $z \in A$ such that 
    $0 < d(y, z) < \omega$. By the triangle inequality, we see that:
    \begin{align*}
        d(x,z) &\leq d(x,y) + d(y,z) \\
        &<  \frac{\epsilon}{2} + \omega \\
        &< \frac{\epsilon}{2} + \frac{\epsilon}{2} \\
        &= \epsilon
    \end{align*}
    So by definition, $x$ is a limit point of $A$. Thus, $x \in A'$. 
    By universal generalization, all limit points of $A'$ are contained in $A'$.
    
    In all cases, every limit point of $A'$ is contained in $A'$. Thus 
    $A'$ is closed.
\end{proof}
\item Prove that $A$ and $\overline{A}$ have the same limit points.
\begin{proof}
    Let $A'$ be the set of limit points of $A$ and let $\overline{A}'$ be the
    set of limit points of $\overline{A}$. We aim to show that $\overline{A}' = A'$. \\

    First let's show that $A' \subseteq \overline{A}'$:
    Let $x \in A'$. Then we know that there exists some point $y \in A$ such that 
    $d(x,y) < \epsilon$ for all $\epsilon > 0$. We also know that since $A \subseteq \overline{A}$,
    $y \in \overline A$. Then we know that there exists some point $y \in \overline{A}$ such that 
    $d(x,y) < \epsilon$ for all $\epsilon > 0$. So y is a limit point of $\overline{A}$. Thus, $y \in \overline{A}'$. \\
    
    Now let's show that $\overline{A}' \subseteq A'$:
    Let $x \in \overline{A}'$. Then we know that for all $\epsilon > 0$ there exists some
    $y \in \overline{A}$ such that $d(x,y) < \epsilon$. We know that $\overline{A} = A \cup A'$.
    So either $y \in A$ or $y \in A'$. Consider these cases separately: \\
    
    Case 1: Assume $y \in A$.
    Then for all $\epsilon > 0$ there exists some
    $y \in A$ such that $d(x,y) < \epsilon$. Thus, $x$ is a limit point of $A$: $x \in A'$. \\
    
    Case 2: Assume $y \in A'$.
    Then for all $\epsilon > 0$ there exists some $y \in A'$ such that $d(x,y) < \frac{\epsilon}{2}$.
    Then choose some $r > 0$ to be equal to $ \frac{\epsilon}{2} - d(x,y)$. Since $y$ is a limit point of $A$, 
    there must be some $z \in A$ such that $z \in N_r(y)$. By the triangle inequality, we have:
    \begin{align*}
        d(x, z) &\leq d(x,y) + d(y,z) \\
        &< \frac{\epsilon}{2} + r \\
        &< \frac{\epsilon}{2} + \frac{\epsilon}{2} \\
        &= \epsilon
    \end{align*}

    Thus, every neighborhood about $x$ contains a $z \in A$. So $x$ is a limit point of $A$: $x \in A'$.

    By universal generalization, $\overline{A}' \subseteq A'$. We've shown the containment both ways, so
    $\overline{A}' = A'$.
\end{proof}
\item Do $A$ and $A'$ always have the same limit points? Prove your claim.

Claim: $A$ and $A'$ do not always have the same limit points.\\

Consider 
\[ A=\left\{ 1+\frac{1}{m} : m \in \mathbb{N} \right\}. \]
We've shown that $A' = {1}$. But since $A'$ is finite, $A'$ cannot have any limit points.
Thus $A'' = \phi$ and $A' \neq A''$.
\end{enumerate}


\end{enumerate}
\end{document} 