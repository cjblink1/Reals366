\documentclass{amsart}
    \usepackage{amssymb}
    \usepackage{enumerate}
    \usepackage[headheight=12pt,textwidth=7in,top=1in, bottom=1in]{geometry}
    
    %some custom commands you may find useful
    
    \usepackage{xparse}
    \DeclareDocumentCommand{\diff}{O{} m}{
        \frac{\mathrm{d} #1}{\mathrm{d}#2}
        }
    \DeclareDocumentCommand{\pdiff}{O{} m}{
        \frac{\partial #1}{\partial #2}
        }
    \DeclareDocumentCommand{\integral}{O{} O{} m O{x}}{
        \int_{#1}^{#2} #3\ \mathrm{d}#4
        }
    
    \def\name{Connor Boyle} %your name goes here
    \def\CM{2239} %your cm goes here
    \def\hwnum{8} %homework number goes here
    
    %these packages create the footer and page numbering
    
    \usepackage{fancyhdr}
    \usepackage{lastpage}
    
    \newtheorem{theorem}{Theorem}
    \newtheorem{lemma}[theorem]{Lemma}
    
    \pagestyle{fancy}
    \lhead{\name}
    \chead{MA 366: Homework Set \hwnum}
    \rhead{\CM}
    \fancyfoot[C]{\footnotesize Page \thepage\ of \pageref{LastPage}}
    
    \fancypagestyle{firststyle}
    {  \renewcommand{\headrulewidth}{0pt}%
       \fancyhf{}%
       \fancyfoot[C]{\footnotesize Page \thepage\ of \pageref{LastPage}}
    }
    
    \begin{document}
    \noindent
    \thispagestyle{firststyle}
    \begin{tabular}{l}
    {\LARGE \textbf{MA 366: Real Analysis} }\\
    {\Large Homework Set \hwnum}
    \end{tabular} \hfill \begin{tabular}{r}
                            \name \\
                            \CM
                            \end{tabular}
    
    \noindent \hrulefill \\\\
    \begin{enumerate}[1.]
    \item Suppose $f: A \to \mathbb{R}$ with $f'$ continuous on $[a,b] \subseteq A$. Prove the following: for all $\epsilon >0$, there exists $\delta >0$ such that if
    \[ 0 < | x-y | < \delta,\]
    then
    \[ \left| \frac{f(x)-f(y)}{x-y} - f'(x) \right| < \epsilon.\]
    %-------------- Problem 1 -------------------
    \begin{proof}
        Note: since $f'$ is a continuous function on $[a,b]$, then by Rudin Thm. 4.19, $f'$ is uniformly continuous
        on $[a,b]$. Thus, for any $t,x \in [a,b]$ we see that for all $\epsilon > 0$ there exists a $\delta > 0$ 
        (dependent only on $\epsilon$) such that
        \[
            0 < |t - x| < \delta \Rightarrow |f'(t) - f'(x)| < \epsilon.
        \] 
        Let $x,y \in [a,b]$. Since $f'$ is continuous, $f$ is differentiable. Thus, the Mean Value Theorem yields a value $t$ between $x$ and $y$ such that
        \[
            f'(t) = \frac{f(x)-f(y)}{x-y}.
        \]
        Fix $\epsilon > 0$. Since $f'$ is uniformly continuous, suppose that 
        \[
            0 < |t - x| \leq |x - y| < \delta \Rightarrow |f'(t) - f'(x)| < \epsilon. 
        \]
        Now consider
        \begin{align*}
            \left| \frac{f(x)-f(y)}{x-y} - f'(x) \right| &= |f'(t) - f'(x)| \\
            &< \epsilon.
        \end{align*}
        Thus, for any $x,y \in [a,b]$,
        \[
            0 < |x - y| < \delta \Rightarrow \left| \frac{f(x)-f(y)}{x-y} - f'(x) \right| < \epsilon.
        \]
    \end{proof}
    
    \newpage
    
    \item Suppose $f: \mathbb{R} \to \mathbb{R}$ such that $f''(x) >0$ for all $x \in (a,b)$. Prove the following: for any $x_0 \in (a,b)$,
    \[ f(x) \geq f(x_0) + f'(x_0)(x-x_0).\]
    %-------------- Problem 2 -------------------
    \begin{proof}
        Since $f''(x) > 0$ for all $x \in (a,b)$, by Rudin Thm. 5.11a, $f'$ is monotonically increasing on $(a,b)$.
        So for any $c,d \in (a,b)$, we know that if $c < d$, then $f'(c) \leq f'(d)$.
        Consider the following cases:
        \begin{itemize}
            \item $x_0,x \in (a,b)$ such that $x_0 < x$:

            By the Mean Value Theorem, we know that there exists a $t \in (x_0, x)$ such that
            \[
                \frac{f(x) - f(x_0)}{x - x_0} = f'(t)
            \]
            Since $f$ is monotonically increasing,
            \begin{align*}
                 f'(x_0) &< f'(t) \\
                 f'(x_0) &< \frac{f(x) - f(x_0)}{x - x_0} \\
                 f'(x_0)\cdot (x - x_0) &< f'(x) - f'(x_0) \\
                 f(x) &> f(x_0) + f'(x_0)\cdot (x - x_0).
            \end{align*}

            \item $x_0,x \in (a,b)$ such that $x_0 = x$:

            Then since $f$ is a function, $f(x) = f(x_0) + f'(x_0)\cdot 0 = f(x_0)$.

            \item $x_0,x \in (a,b)$ such that $x_0 > x$:

            By the Mean Value Theorem, we know that there exists a $t \in (x, x_0)$ such that
            \[
                \frac{f(x) - f(x_0)}{x - x_0} = f'(t)
            \]
            Since $f$ is monotonically increasing,
            \begin{align*}
                 f'(x_0) &> f'(t) \\
                 f'(x_0) &> \frac{f(x) - f(x_0)}{x - x_0} \\
                 f'(x_0)\cdot (x - x_0) &< f'(x) - f'(x_0) \\
                 f(x) &> f(x_0) + f'(x_0)\cdot (x - x_0).
            \end{align*}
        \end{itemize}
        In all cases, 
        \[ f(x) \geq f(x_0) + f'(x_0)(x-x_0).\]
    \end{proof}

    
    \newpage
    \item Let $\alpha: \mathbb{R} \to \mathbb{R}$ be the \emph{greatest integer function}: $\alpha(x):= \sup \{ a \in \mathbb{Z}: a \leq x\}$. Let $f:[0,\infty) \to \mathbb{R}$ be a bounded function that is continuous at all $n \in \mathbb{N}$. Compute the value of
    \[ \int_0^n f\ \mathrm{d}\alpha.\]
    Prove your assertion.
    %-------------- Problem 3 -------------------
    \begin{theorem}
        Let $\alpha: \mathbb{R} \to \mathbb{R}$ be the \emph{greatest integer function}: $\alpha(x):= \sup \{ a \in \mathbb{Z}: a \leq x\}$. Let $f:[0,\infty) \to \mathbb{R}$ be a bounded function that is continuous at all $n \in \mathbb{N}$. Then 
        \[
            \int_0^n f\ \mathrm{d}\alpha = \sum_{k = 1}^{n} f(k)
        \]
        \begin{proof}
            Note: the greatest integer function can be expressed as
            \[
                \alpha(x) = -1 + \sum_{k = 1}^\infty I(x - k)    
            \]
            Where $I(x)$ is the unit step function. As such we can say that 
            \[ \Delta \alpha_i = \begin{cases} 1 & [x_{i-1}, x_i] \cap \mathbb{N} \neq \phi \\
                0 & [x_{i-1}, x_i] \cap \mathbb{N} = \phi.
                \end{cases}\]
            Consider a partition $P$ of $[0,n]$ such that every $[x_{i-1}, x_i]$ in the partition contains
            exatcly one natural number.
            Then 
            \begin{align*}
                U(P,f,\alpha) = \sum_{x_i\in P} \sup_{[x_{i-1}, x_i]} f(x) \cdot 1 \\
                L(P,f,\alpha) = \sum_{x_i\in P} \inf_{[x_{i-1}, x_i]} f(x) \cdot 1 \\
            \end{align*}
            As $P$ becomes more refined, 
            \begin{align*}
                \sum_{x_i\in P} \sup_{[x_{i-1}, x_i]} f(x) \to \sum_{k=1}^n f(k) \\ 
                \sum_{x_i\in P} \inf_{[x_{i-1}, x_i]} f(x) \to \sum_{k=1}^n f(k). \\
            \end{align*}
            Thus,
            \[
                \int_0^n f\ \mathrm{d}\alpha = \sum_{k = 1}^{n} f(k)    
            \]
        \end{proof}
    \end{theorem}

  
    \newpage
    \item Consider the function $f: \mathbb{R} \to \mathbb{R}$ defined by
    \[ f(x) = \begin{cases} 1 & x \in \mathbb{Q} \\
                            0 & x \not \in \mathbb{Q}.
                            \end{cases}\]
    Is $f$ Riemann integrable over any non-trivial interval $[a,b]$? Prove your claim.
    %-------------- Problem 4 -------------------
    \begin{theorem}
        $f$ is not Riemann integrable over any non-trivial interval $[a,b]$. 
        \begin{proof}
            Consider a partition $P$ of $[a,b]$. We would like to investigate 
            \[
                U(P,f) - L(P,f)    
            \]
            Which is equivalent to 
            \[
                \sum_{x_i \in P} \left( \sup_{[x_{i-1}, x_i]} f(x) - \inf_{[x_{i-1}, x_i]} f(x) \right) \cdot \Delta x_i
            \]
            Since the rationals are dense in the reals, for any $[x_{i-1}, x_i]$,
            \begin{align*}
                \sup_{[x_{i-1}, x_i]} f(x) = 1 \\
                \inf_{[x_{i-1}, x_i]} f(x) = 0
            \end{align*}
            Thus, $U(P,f) - L(P,f)$ cannot be made any smaller than $b - a$. 
        \end{proof}
    \end{theorem}
    
    
    \newpage
    \item  Suppose $f: [a,b] \to \mathbb{R}$ satisfies the following conditions:
    \begin{itemize}
    \item $f$ is continuous on $[a,b]$,
    \item $f(x) \leq -6$ for all $x \in [a,b]$, and
    \item $\displaystyle \int_a^b f\ \mathrm{d}x = 6a - 6b$.
    \end{itemize}
    Prove that $f(x)=-6$ for all $x \in [a,b]$.
    %-------------- Problem 5 -------------------
    
 
    \newpage
    \item Let $f: [a,b] \to \mathbb{R}$ be a continuous function, and let $\alpha: [a,b] \to \mathbb{R}$ be monotone increasing. Show that there exists $\xi \in [a,b]$ such that
    \[ \int_a^b f\ \mathrm{d}\alpha = f(\xi) \cdot \big( \alpha(b) - \alpha(a) \big).\]
    %-------------- Problem 6 -------------------


    \end{enumerate}
    \end{document} 