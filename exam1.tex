\documentclass{amsart}
\usepackage{amssymb}
\usepackage{enumerate}
\usepackage[headheight=12pt,textwidth=7in,top=1in, bottom=1in]{geometry}

%some custom commands you may find useful

\usepackage{xparse}
\DeclareDocumentCommand{\diff}{O{} m}{
    \frac{\mathrm{d} #1}{\mathrm{d}#2}
    }
\DeclareDocumentCommand{\pdiff}{O{} m}{
    \frac{\partial #1}{\partial #2}
    }
\DeclareDocumentCommand{\integral}{O{} O{} m O{x}}{
    \int_{#1}^{#2} #3\ \mathrm{d}#4
    }

\def\name{Connor Boyle} %your name goes here
\def\CM{2239} %your cm goes here
\def\hwnum{5} %homework number goes here

%these packages create the footer and page numbering

\usepackage{fancyhdr}
\usepackage{lastpage}

\newtheorem{theorem}{Theorem}
\newtheorem{lemma}[theorem]{Lemma}

\pagestyle{fancy}
\lhead{\name}
\chead{MA 366: Homework Set \hwnum}
\rhead{\CM}
\fancyfoot[C]{\footnotesize Page \thepage\ of \pageref{LastPage}}

\fancypagestyle{firststyle}
{  \renewcommand{\headrulewidth}{0pt}%
   \fancyhf{}%
   \fancyfoot[C]{\footnotesize Page \thepage\ of \pageref{LastPage}}
}

\begin{document}
\noindent
\thispagestyle{firststyle}
\begin{tabular}{l}
{\LARGE \textbf{MA 366: Real Analysis} }\\
{\Large Homework Set \hwnum}
\end{tabular} \hfill \begin{tabular}{r}
                        \name \\
                        \CM
                        \end{tabular}

\noindent \hrulefill \\\\
First, here are some useful lemmas:
%------------ Lemmas -------------------

\begin{enumerate}[1.]
\item Find, and give a clean $\varepsilon$--$N$ proof for, the limit of $\displaystyle  q_n=\frac{2n^2+3n}{7n^2-13}$ as $n \to \infty$.
%-------------- Problem 1 -------------------


\newpage

\item  \begin{enumerate}[(a)]
	\item (10 points) Find the supremum and infimum of the following set.
	Clearly prove your claims.
	\[ A = \left\{ \frac{m}{n} : m,n \in \mathbb N,\ 2m<5n \right\}. \]
	\item (10 points) Given a set $B$, define the set $\dfrac{1}{B} := \left\{\dfrac{1}{x} : x \in B \right\}$. If $B$ is a subset of the positive real numbers, prove that
	\[ \sup \left(\frac{1}{B} \right) = \begin{cases}
                                            \dfrac{1}{\inf B} & \inf(B) > 0 \\
	                                                          & \\
	                                        \infty            & \inf(B) = 0.
	\end{cases}
	\]
Note: We interpret $\sup E = \infty$ to mean that $E$ is an unbounded set.
\end{enumerate}
%-------------- Problem 2 -------------------

\newpage
\item Consider the following metric on $\mathbb R$
\[ d(x,y)=\frac{|x-y|}{1+|x-y|}. \]
Prove that: $U$ is an open set in $\mathbb R$ with the metric $d$ if and only if $U$ is an open
set in $\mathbb R$ with the Archimedean metric $d_{\infty}(x,y)=|x-y|$. You have already proven in your homework that $d$ is a metric, you do \textbf{not} need to repeat that here.

%-------------- Problem 3 -------------------
\newpage
\item Suppose $f: A \to B$ is an onto map. Prove: if $B$ is uncountable, then $A$ is uncountable .

%-------------- Problem 4 -------------------
\newpage
\item In a metric space $X$ with metric $d$, we define the ``edge of $A$" to be the set
\[ e(A)=\{ x\in X : \forall r >0, \quad N_r(x)\cap A\neq \emptyset \quad \text{and} \quad N_r(x)\cap A^c \neq \emptyset   \}.\]
\begin{enumerate}[(a)]
	\item (10 points) Prove that $e(A)$ is closed in $X$.
	\item (10 points) Prove that $A$ is closed if and only if $e(A)\subset A$.
\end{enumerate}

%-------------- Problem 5 -------------------
\newpage
\item Let $(x_n)$ be a sequence of real numbers. We say that $(x_n)$ is \emph{mostly-Cauchy} if
\[ |x_n - x_{n+1}| \xrightarrow{n \to \infty} 0.\]
Note that mostly-Cauchy sequences need not converge. For example, $x_n=\log n$ is mostly-Cauchy and divergent.

Now, suppose $(x_n)$ is a mostly-Cauchy sequence that additionally has the property that for all $n \in \mathbb{N}$:
\[ x_{2n} \leq x_{2n+2} \qquad \text{and} \qquad x_{2n+1} \leq x_{2n-1}.\]
Prove that the following results hold for such a sequence:
\begin{enumerate}[(a)]
	\item (5 points) The subsequence $x_{2n}$ is bounded above.
	\item (5 points) The subsequence $x_{2n+1}$ is bounded below.
	\item (10 points) The sequence $(x_n)$ converges.
\end{enumerate}
%-------------- Problem 6 -------------------
\end{enumerate}
\end{document} 