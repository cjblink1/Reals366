\documentclass{amsart}
\usepackage{amssymb}
\usepackage{enumerate}
\usepackage[headheight=12pt,textwidth=7in,top=1in, bottom=1in]{geometry}

%some custom commands you may find useful

\usepackage{xparse}
\DeclareDocumentCommand{\diff}{O{} m}{
    \frac{\mathrm{d} #1}{\mathrm{d}#2}
    }
\DeclareDocumentCommand{\pdiff}{O{} m}{
    \frac{\partial #1}{\partial #2}
    }
\DeclareDocumentCommand{\integral}{O{} O{} m O{x}}{
    \int_{#1}^{#2} #3\ \mathrm{d}#4
    }

\def\name{Connor Boyle} %your name goes here
\def\CM{2239} %your cm goes here
\def\hwnum{1} %homework number goes here

%these packages create the footer and page numbering

\usepackage{fancyhdr}
\usepackage{lastpage}

\newtheorem{theorem}{Theorem}
\newtheorem{lemma}[theorem]{Lemma}

\pagestyle{fancy}
\lhead{\name}
\chead{MA 366: Exam \hwnum}
\rhead{\CM}
\fancyfoot[C]{\footnotesize Page \thepage\ of \pageref{LastPage}}

\fancypagestyle{firststyle}
{  \renewcommand{\headrulewidth}{0pt}%
   \fancyhf{}%
   \fancyfoot[C]{\footnotesize Page \thepage\ of \pageref{LastPage}}
}

\begin{document}
\noindent
\thispagestyle{firststyle}
\begin{tabular}{l}
{\LARGE \textbf{MA 366: Real Analysis} }\\
{\Large Exam \hwnum}
\end{tabular} \hfill \begin{tabular}{r}
                        \name \\
                        \CM
                        \end{tabular}

\noindent \hrulefill \\\\
\begin{enumerate}[1.]
\item Find, and give a clean $\varepsilon$--$N$ proof for, the limit of $\displaystyle  q_n=\frac{2n^2+3n}{7n^2-13}$ as $n \to \infty$.
%-------------- Problem 1 -------------------
\begin{lemma}
	For $n \geq 26$, $22n \geq 21n + 26$.
	\begin{proof}
		Consider:
		\begin{align*}
			n &\geq 26 \\
			21n + n &\geq 21n + 26 \\
			22n &\geq 21n + 26
		\end{align*}
	\end{proof}
\end{lemma}
\begin{lemma}
	For $n\geq 2$, $22n^2 \leq 49n^2 - 91$.
	\begin{proof}
		Consider:
		\begin{align*}
			n &\geq 2 \\
			n^2 &\geq 4 \\
			27n^2 &\geq 108 \\
			-27n^2 &\leq -108 \\
			-27n^2 &\leq -91 \\
			49n^2 -27n^2 &\leq 49n^2 -91 \\
			22n^2 &\leq 49n^2 -91
		\end{align*}
	\end{proof}
\end{lemma}
\begin{lemma}
	For $n\geq 2$, 
	\[
		\frac{21n + 26}{49n^2 -91} > 0	
	\]
	\begin{proof}
		Consider:
		\begin{align*}
			n &\geq 2 \\
			n &> -2 \\
			13n &> -26 \\
			21n &> -26 \\
			21n + 26 &> 0
		\end{align*}
		Also consider:
		\begin{align*}
			n &\geq 2 \\
			n^2 &\geq 4 \\
			n^2 &> 2 \\
			49n^2 &> 98 \\
			49n^2 &> 91 \\
			49n^2 -91 &> 0
		\end{align*}
		Since 
		\[
			x > 0 \text{ and } y > 0 \implies \frac{x}{y} > 0	
		\]
		Then for $n \geq 2$,
		\[
			\frac{21n + 26}{49n^2 -91} > 0	
		\]
	\end{proof}
\end{lemma}
\begin{theorem}
	$\lim_{n \to \infty} \frac{2n^2+3n}{7n^2-13} = \frac{2}{7}$.
	\begin{proof}
		Fix an $\epsilon > 0$. Now, by the archimedian principle, choose a natural number $N$ such that
		\[
			N\cdot\epsilon > 1 \text{ and } N \geq\max\{2,2,26\} = 26
		\]
		Suppose that $n \geq N$. Then:
		\begin{align*}
			\frac{1}{N} &< \epsilon \\
			\frac{1}{n} &< \\
			\frac{22n}{22n^2} &= \\
			\frac{21n + 26}{22n^2} &< \text{ by Lemma 1 }\\ 
			\frac{21n + 26}{49n^2 - 91} &< \text{ by Lemma 2 } \\
			\left|\frac{21n + 26}{49n^2 - 91}\right| &= \text{ by Lemma 3 }\\
			\left|\frac{14n^2 + 21n - 14n^2 + 26}{49n^2 - 91}\right| &= \\
			\left|\frac{14n^2 + 21n}{49n^2 - 91} - \frac{14n^2 - 26}{49n^2 - 91}\right| &= \\
			\left|\frac{2n^2 + 3n}{7n^2 - 13} - \frac{2}{7}\right| &= \\
			d\left(q_n, \frac{2}{7}\right) &=
		\end{align*}
	\end{proof}
\end{theorem}

\newpage

\item  \begin{enumerate}[(a)]
	\item (10 points) Find the supremum and infimum of the following set.
	Clearly prove your claims.
	\[ A = \left\{ \frac{m}{n} : m,n \in \mathbb N,\ 2m<5n \right\}. \]
	\item (10 points) Given a set $B$, define the set $\dfrac{1}{B} := \left\{\dfrac{1}{x} : x \in B \right\}$. If $B$ is a subset of the positive real numbers, prove that
	\[ \sup \left(\frac{1}{B} \right) = \begin{cases}
                                            \dfrac{1}{\inf B} & \inf(B) > 0 \\
	                                                          & \\
	                                        \infty            & \inf(B) = 0.
	\end{cases}
	\]
Note: We interpret $\sup E = \infty$ to mean that $E$ is an unbounded set.
\end{enumerate}
%-------------- Problem 2 -------------------
\begin{lemma}
	\[
		\frac{5n - 2}{2n} < \frac{5}{2}, n \in \mathbb{N}	
	\]
	\begin{proof}
		Consider:
		\begin{align*}
			\frac{5n - 2}{2n} &= \frac{5n}{2n} - \frac{2}{2n} \\
			&= \frac{5}{2} - \frac{1}{n} \\
			&< \frac{5}{2} \text { since } \frac{1}{n} > 0 \text{ for all } n\in \mathbb{N}.
		\end{align*}
	\end{proof}
\end{lemma}
\begin{lemma}
	If $n\cdot \epsilon > 1$, then
	\[
		\frac{5n - 2}{2n} > \frac{5}{2} - \epsilon, n \in \mathbb{N}	
	\]
	\begin{proof}
		Consider:
		\begin{align*}
			\frac{5n - 2}{2n} &= \frac{5n}{2n} - \frac{2}{2n} \\
			&= \frac{5}{2} - \frac{1}{n} \\
			&> \frac{5}{2} - \epsilon \text { since } \frac{1}{n} < \epsilon.
		\end{align*}
	\end{proof}
\end{lemma}
\begin{lemma}
	\[
		\frac{5n - 2}{2n} \in A \text{ for all } n \in \mathbb{N}
	\]
	\begin{proof}
		Note: if $m/n \in A$, then $2m < 5n$.\\
		Consider:
		\begin{align*}
			2(5n - 2) &= 10n - 4 \\
			&< 10n \\
			&= 5(2n)
		\end{align*}
		Thus, 
		\[
			\frac{5n - 2}{2n} \in A \text{ for all } n \in \mathbb{N}
		\]
	\end{proof}
\end{lemma}
\begin{lemma}
	\[
		\frac{1}{n} \in A \text{ for all } n \in \mathbb{N}	
	\]
	\begin{proof}
		Note: if $m/n \in A$, then $2m < 5n$.\\
		Consider:
		\begin{align*}
			2(1) &= 2 \\
			&< 5 \\
			&= 5(1) \\
			&\leq 5(n) \text{ for all } n \in \mathbb{N}.
		\end{align*}
	\end{proof}
\end{lemma}
\newpage
\begin{theorem}
	$\sup(A) = \frac{5}{2}$
	\begin{proof}
		First, we aim to show that $\frac{5}{2}$ is an upper bound for $A$. Consider
		any arbitrary element $\frac{m}{n} \in A$. We know that:
		\begin{align*}
			2m &< 5n \\
			\frac{2m}{n} &< 5 \\
			\frac{m}{n} &< \frac{5}{2} \\
		\end{align*}
		Thus, by the definition of order of $\mathbb{Q}$, $\frac{5}{2}$ is an upper bound of $A$.
		It remains to show that $\frac{5}{2}$ is the least upper bound of $A$. Let $\epsilon > 0$. 
		Then by the archimedian principle, choose an $n\in N$ such that $n\cdot\epsilon > 1$.
		Consider the following $\beta_n \in\mathbb{Q}$:
		\[
			\beta_n = \frac{5n - 2}{2n}
		\]
		By Lemma 5 and Lemma 6, we know that:
		\[
			\frac{5}{2} - \epsilon < \beta_n < \frac{5}{2}
		\]
		and by Lemma 7, we know that $\beta_n \in A$.
		Thus, we see that $\frac{5}{2}$ is the least upper bound of $A$.
	\end{proof}
\end{theorem}
\begin{theorem}
	$\inf(A) = 0$
	\begin{proof}
		First, we aim to show that $0$ is a lower bound for $A$. Consider any arbitrary element
		$\frac{m}{n}\in A$. We know that:
		\begin{align*}
			0 &< m \text{ since } m \in \mathbb{N}\\
			0 &< \frac{m}{n} \text{ since } n \in \mathbb{N} \\
		\end{align*}
		Thus, by definition of order of $\mathbb{Q}$, $0$ is a lower bound of $A$. It remains to show that
		$0$ is the greatest lower bound of $A$. Let $\epsilon > 0$. Then, by the archimedian principle,
		choose an $n\in\mathbb{N}$ such that $n\cdot\epsilon > 1$. Consider the following $\gamma_n \in \mathbb{Q}$:
		\[
			\gamma_n = \frac{1}{n}
		\]
		We see that 
		\[
			0 < \frac{1}{n} < \epsilon \text{ by construction, }	
		\]
		and that $\gamma_n \in A$ by Lemma 8. Hence, $0$ is the greatest lower bound of $A$.
	\end{proof}
\end{theorem}
\begin{theorem}
	If $\inf(B) > 0$, then 
	\[
		\sup\left(\frac{1}{B}\right) = \frac{1}{\inf(B)}
	\]
	\begin{proof}
		Note: By assumption, $B$ is non-empty. \\
		Let $\inf(B) = \beta > 0$.We aim to show that 
		\[
			\sup\left(\frac{1}{B}\right) = \frac{1}{\beta}.	
		\]
		Consider any element $x \in B$ and its corresponding element $y = \frac{1}{x} \in \frac{1}{B}$. We know that 
		\begin{align*}
			\beta &\leq x \text{ by definition of infimum } \\
			\frac{\beta}{x} &\leq 1 \text{ since } x \in B \subset \mathbb{R^+} \\
			\frac{1}{x} &\leq \frac{1}{\beta} \text{ since } \beta > 0 \\
			y &\leq \frac{1}{\beta} \text{ for all } x \in B
		\end{align*}
		Thus, $\frac{1}{\beta}$ is an upper bound for $\frac{1}{B}$.
		It remains to show that $\frac{1}{\beta}$ is the least upper bound of $\frac{1}{B}$.
		Note: $\beta$ is the greatest lower bound of $B$. Therefore, for any $\epsilon > 0$, there exists an
		$\alpha \in B$ such that
		\[
			\beta \leq \alpha < \beta + \epsilon	
		\]
		As such, fix $\epsilon > 0$. Then:
		\begin{align*}
			\beta &\leq \alpha < \beta + \epsilon \\
			\frac{1}{\beta + \epsilon} &< \frac{1}{\alpha} \leq \frac{1}{\beta} \\
			\frac{1}{\beta} - \frac{\epsilon}{\beta^2 + \epsilon\beta} &= \\
		\end{align*}
		Thus, by choosing $\epsilon$ arbitrarily close to zero, we can assert that there exists
		$\frac{1}{\alpha} \in B$ such that:
		\[
			\frac{1}{\beta} - \frac{\epsilon}{\beta^2 + \epsilon\beta} < \frac{1}{\alpha} \leq \frac{1}{\beta}.
		\]
		It follows that $\frac{1}{\beta}$ is the least upper bound of $\frac{1}{B}$.
	\end{proof}
\end{theorem}
\begin{theorem}
	If $\inf(B) = 0$, then 
	\[
		\sup\left(\frac{1}{B}\right) = +\infty \text{ (i.e. $\frac{1}{B}$ is not bounded above) }
	\]
	\begin{proof}
		Note: since $\inf(B) = 0$ and $B \subseteq \mathbb{R^+}$, for all $\epsilon > 0$, there are infinitely many $\alpha_n \in B$ such that
		\[
			\epsilon > \alpha_1 > \alpha_2 > ... > 0
		\]
		Fix $\epsilon > 0$. Then:
		\begin{align*}
			\epsilon > \alpha_1 > &\alpha_2 > ... > 0 \\
			\frac{1}{\epsilon} < \frac{1}{\alpha_1} < &\frac{1}{\alpha_2} < ... < \frac{1}{0} = +\infty \\
		\end{align*}
		So we have shown that for every $\frac{1}{\alpha_n} \in \frac{1}{B}$, there exists a $\frac{1}{\alpha_{n+1}} \in \frac{1}{B}$ such that
		\[
			\frac{1}{\alpha_n} < \frac{1}{\alpha_{n+1}}
		\]
		Therefore, $\frac{1}{B}$ is not bounded above.
	\end{proof}
\end{theorem}

\newpage
\item Consider the following metric on $\mathbb R$
\[ d(x,y)=\frac{|x-y|}{1+|x-y|}. \]
Prove that: $U$ is an open set in $\mathbb R$ with the metric $d$ if and only if $U$ is an open
set in $\mathbb R$ with the Archimedean metric $d_{\infty}(x,y)=|x-y|$. You have already proven in your homework that $d$ is a metric, you do \textbf{not} need to repeat that here.

%-------------- Problem 3 -------------------
We will use the notation
		\[
			N_{d,\epsilon}(x) \text{ to mean } \{ y: d(x, y) < \epsilon\}	
		\]
\begin{lemma}
	If $a \geq 0$, then 
	\[
		\frac{a}{1+a} \leq a
	\]
	\begin{proof}
		Consider:
		\begin{align*}
			a &\geq 0 \\
			a^2 &\geq 0 \\
			a + a^2 &\geq a \\
			a(1 + a) &\geq a \\
			a &\geq \frac{a}{1 + a} \\
			\frac{a}{1 + a} &\leq a
		\end{align*}
	\end{proof}
\end{lemma}
\begin{lemma}
	\[
		N_{d_\infty,\epsilon}(x) \subseteq N_{d,\epsilon}(x)	
	\]
	\begin{proof}
		Let $y \in N_{d_\infty,\epsilon}(x)$. Then
		\[
			|x - y| < \epsilon	
		\]
		But we know from Lemma 13 that
		\[
			\frac{|x - y|}{1 + |x - y|} \leq |x - y|	
		\]
		So
		\begin{align*}
			\frac{|x - y|}{1 + |x - y|} < \epsilon \\
			y \in N_{d,\epsilon}(x)
		\end{align*}
	\end{proof}
\end{lemma}
\begin{lemma}
	\[
		N_{d,\epsilon'}(x) \subseteq N_{d_\infty, \epsilon}(x) \text { where } \epsilon' = \frac{\epsilon}{1 + \epsilon}.	
	\]
	\begin{proof}
		Let $y \in N_{d,\epsilon'}(x)$. Then
		\begin{align*}
			\frac{|x - y|}{1 + |x - y|} &< \frac{\epsilon}{1 + \epsilon} = \epsilon' \\
			|x - y|\cdot(1 + \epsilon) &< (1 + |x - y|)\cdot \epsilon \\
			|x - y| + |x - y|\cdot \epsilon &< \epsilon + |x - y|\cdot \epsilon \\
			|x - y| &< \epsilon
		\end{align*}
		Thus, $y \in N_{d_\infty, \epsilon}(x)$.
	\end{proof}
\end{lemma}
\newpage
% \begin{theorem}
% 	If $U$ is an open set in $(\mathbb{R},d)$, then $U$ is an open set in $(\mathbb{R},d_\infty)$.
% 	\begin{proof}
% 		We will use the notation
% 		\[
% 			N_{d,\epsilon}(x) \text{ to mean } \{ y: d(x, y) < \epsilon\}	
% 		\]
% 		Note: $U$ is an open set in $(\mathbb{R},d)$ means that for every $x \in U$ there exists an
% 		$\epsilon > 0$ such that $N_{d,\epsilon}(x) \subseteq U$.
% 		First, assume $U$ is an open set in $(\mathbb{R},d)$. Then pick an arbitrary $x \in U$ and fix $\epsilon > 0$ such that
% 		\[
% 			N_{d,\epsilon}(x) \subseteq U
% 		\]
% 		Take any element $y \in N_{d,\epsilon}(x)$. Then
% 		\[
% 			\frac{|x - y|}{1 + |x - y|} < \epsilon	
% 		\]
% 		But we know from Lemma 13 that
% 		\[
% 			\frac{|x - y|}{1 + |x - y|} \leq |x - y|
% 		\]
% 		So 
% 		\begin{align*}
% 			\frac{|x - y|}{1 + |x - y|} < \epsilon \\
% 			y \in N_{d,\epsilon}(x)
% 		\end{align*}
% 		Thus, $N_{d_\infty,\epsilon}(x) \subseteq N_{d,\epsilon}(x) \subseteq U$. So 
% 		By universal generalization, $U$ is an open set in $(\mathbb{R},d)$.
% 	\end{proof}
% \end{theorem}
\begin{theorem}
	If $U$ is an open set in $(\mathbb{R},d_\infty)$, then $U$ is an open set in $(\mathbb{R},d)$.
	\begin{proof}
		Note: $U$ is an open set in $(\mathbb{R},d_\infty)$ means that for every $x \in U$ there exists an
		$\epsilon > 0$ such that $N_{d_\infty,\epsilon}(x) \subseteq U$.
		First, assume $U$ is an open set in $(\mathbb{R},d_\infty)$. Then pick an arbitrary $x \in U$ and fix $\epsilon > 0$ such that
		\[
			N_{d_\infty,\epsilon}(x) \subseteq U
		\]
		We know from Lemma 15 that
		\[
			N_{d,\epsilon'}(x) \subseteq N_{d_\infty,\epsilon}(x) \text{ where } \epsilon' = \frac{\epsilon}{1 + \epsilon}
		\]
		So for any $\epsilon > 0$ there exists an $\epsilon' > 0$ such that
		\begin{align*}
			N_{d,\epsilon'}(x) \subseteq N_{d_\infty,\epsilon}(x) \subseteq U \\
			N_{d,\epsilon'}(x) \subseteq U
		\end{align*}
		By universal generalization, for all $x \in U$, there exists an $\epsilon' > 0$ such that 
		\[
			N_{d,\epsilon'}(x) \subseteq U
		\]
		Thus, $U$ is an open set in $(\mathbb{R},d)$.
	\end{proof}
\end{theorem}
\begin{theorem}
	If $U$ is an open set in $(\mathbb{R},d)$, then $U$ is an open set in $(\mathbb{R},d_\infty)$.
	\begin{proof}
		First, assume $U$ is an open set in $(\mathbb{R},d)$. Then pick an arbitrary $x \in U$ and fix $\epsilon > 0$ such that
		\[
			N_{d,\epsilon}(x) \subseteq U
		\]
		We know from Lemma 14 that
		\[
			N_{d_\infty,\epsilon}(x) \subseteq N_{d,\epsilon}(x)
		\]
		So for every $\epsilon > 0$ we see that
		\begin{align*}
			N_{d_\infty,\epsilon}(x) \subseteq N_{d,\epsilon}(x) \subseteq U \\
			N_{d_\infty,\epsilon}(x) \subseteq U
		\end{align*}
		By universal generalization, for all $x \in U$, there exists an $\epsilon > 0$ such that 
		\[
			N_{d_\infty,\epsilon}(x) \subseteq U
		\]
		Thus, $U$ is an open set in $(\mathbb{R},d_\infty)$.
	\end{proof}
\end{theorem}
\begin{theorem}
	$U$ is an open set in $\mathbb R$ with the metric $d$ if and only if $U$ is an open
set in $\mathbb R$ with the Archimedean metric $d_{\infty}(x,y)=|x-y|$.
	\begin{proof}
		This follows from Theorem 16 and Theorem 17.
	\end{proof}
\end{theorem}
\newpage
\item Suppose $f: A \to B$ is an onto map. Prove: if $B$ is uncountable, then $A$ is uncountable .
%-------------- Problem 4 -------------------
\begin{lemma}
	If $h: X \to Y$ is injective and $Y$ is countable, then $X$ is countable.
	\begin{proof}
		Let the range of $h$ be $Z$. Since $Z \subseteq Y$ and $Y$ is countable, then $Z$ is at most countable (by Rudin Thm 2.8).
		Then since $h$ is injective, every element $z \in Z$ is associated to a unique element $x \in X$ such that $h(x) = z$. Thus there is a bijection between
		$X$ and $Z$. Since $Z$ is countable, then $X$ is countable.
	\end{proof}
\end{lemma}
\begin{theorem}
	If $B$ is uncountable, then $A$ is uncountable.
	\begin{proof} 
		Note: if $f$ is an onto map, then for every $b\in B$ there exists an $a \in A$ such that
		$f(a) = b$.
		Also note: if $A$ is countable, then there exists a bijection between $\mathbb{N}$ and $A$.
		Consider the contrapositive statement:
		\[
			\emph{If $A$ is countable, then $B$ is countable}.
		\]

		Assume that $A$ is countable. 
		Consider any $b \in B$. Then $f^{-1}(b) \subseteq A$ is at most countable and non-empty. Therefore, we can enumerate $f^{-1}(b)$
		as follows:
		\[
			f^{-1}(b) = \{a_{1,b}, a_{2,b}, a_{3,b}, ... \}	
		\]
		Let us construct a function $g: B \to A$ defined as follows:
		\[
			g(b) = a_{1,b}	
		\]
		We must first show that $g$ is injective.
		Consider the case where $g(b_1) = g(b_2)$ for some $b_1, b_2 \in B$. Then we know that 
		\[
			a_{1,b_1} = a_{1,b_2}
		\]
		Since $f$ is a function, $f^{-1}(b_1) \cap f^{-1}(b_2) = \phi$ (one element of the domain cannot be sent to multiple values in the range).
		Thus, the only way for $a_{1,b_1} = a_{1,b_2}$ is if $b_1 = b_2$. Thus, $g$ is injective.
		Finally, we see from Lemma 19 that, since $g: B \to A$ is injective and $A$ is countable,
		then $B$ is countable.
	\end{proof}
\end{theorem}
\newpage
\item In a metric space $X$ with metric $d$, we define the ``edge of $A$" to be the set
\[ e(A)=\{ x\in X : \forall r >0, \quad N_r(x)\cap A\neq \emptyset \quad \text{and} \quad N_r(x)\cap A^c \neq \emptyset   \}.\]
\begin{enumerate}[(a)]
	\item (10 points) Prove that $e(A)$ is closed in $X$.
	\item (10 points) Prove that $A$ is closed if and only if $e(A)\subset A$.
\end{enumerate}
%-------------- Problem 5 -------------------
\begin{lemma}
	$e(A) \subseteq \overline{A}$ where $\overline{A}$ is the closure of $A$.
	\begin{proof}
		Let $x \in e(A)$. Then we know that
		\[
			\forall r > 0, N_r(x) \cap A \neq \phi	
		\]
		Consider the following cases: \\

		Case 1: $x \in A$. Then since $A \subseteq \overline{A}$, $x \in \overline{A}$. \\
		
		Case 2: $x \notin A$. Then for all $r > 0$, there exists an $a \in A$ such that $d(x, a) < r$ and $x \neq a$.	
		Thus, $x$ is a limit point of $A$, so $x \in A'$. Since $A' \subseteq \overline{A}$, then $x \in \overline{A}$. \\
		
		In every case, $x \in \overline{A}$. By universal generalization, $e(A) \subseteq \overline{A}$
	\end{proof}	
\end{lemma}
\begin{theorem}
	$e(A)$ is closed in $X$.
	\begin{proof}
		Note: if $e(A)$ is closed, then $e(A)$ contains all of its limit points.

		Let $\alpha$ be a limit point of $e(A)$. Then for all $r > 0$, there exists an $x \in e(A)$ such that
		$d(\alpha, x) < \frac{r}{2}$. But since $x \in e(A)$, we know that
		\begin{align*}
			&\text{(1) } \forall r > 0, N_r(x) \cap A \neq \phi \\
			&\text{(2) } \forall r > 0, N_r(x) \cap A^c \neq \phi \\
		\end{align*}
		As such,
		\begin{align*}
			&\text{(1) } \forall r > 0, \exists a \in A \text{ such that } d(x, a) < \frac{r}{2} \\
			&\text{(2) } \forall r > 0, \exists b \in A^c \text{ such that } d(x, b) < \frac{r}{2}. \\
		\end{align*}
		First consider the distance $d(\alpha, a)$. By the triangle inequality, we have:
		\begin{align*}
			d(\alpha, a) &\leq d(\alpha, x) + d(x,a) \\
			&< \frac{r}{2} + \frac{r}{2} \\
			&= r
		\end{align*}
		So for all $r > 0$, $d(\alpha, a) < r$. \\
		Now consider the distance $d(\alpha, b)$. By the triangle inequality, we have:
		\begin{align*}
			d(\alpha, b) &\leq d(\alpha, x) + d(x,b) \\
			&< \frac{r}{2} + \frac{r}{2} \\
			&= r
		\end{align*}
		So for all $r > 0$, $d(\alpha, b) < r$. \\
		These results imply:
		\begin{align*}
			&\text{(1) } \forall r > 0, N_r(\alpha) \cap A \neq \phi \\
			&\text{(2) } \forall r > 0, N_r(\alpha) \cap A^c \neq \phi \\
		\end{align*}
		By definition, $\alpha \in e(A)$. By universal generalization, $e(A)$ contains all of its limit points.
		Therefore $e(A)$ is closed in $X$.
	\end{proof}
\end{theorem}
\newpage
\begin{theorem}
	If $A$ is closed, then $e(A) \subseteq A$.
	\begin{proof}
		Let $A$ be closed. Then, $A = \overline{A}$.
		But we know from Lemma 21 that 
		\[
			e(A) \subseteq \overline{A}	
		\]
		As such,
		\[
			e(A) \subseteq \overline{A} = A.
		\]
		Therefore $e(A) \subseteq A$.
	\end{proof}
\end{theorem}
\begin{theorem}
	If $e(A) \subseteq A$ then $A$ is closed.
	\begin{proof}
		Assume $e(A) \subseteq A$.
		Let $a$ be a limit point of $A$. We aim to show that $a \in A$.
		Consider a neighborhood around $a$ with radius $r$ in the following two cases:

		Case 1: $\exists r > 0$ such that $N_r(a) \subseteq A$. Since $a \in N_r(a)$, then $a \in A$. \\
		
		Case 2: $\nexists r > 0$ such that $N_r(a) \subseteq A$. Then for all $r > 0$, $N_r(a)$ contains at least one element 
		that is not contained in $A$. Thus, $N_r(a) \cap A^c \neq \phi$. But $a$ is also a limit point, so we know that
		for all $r > 0$, $N_r(a) \cap A \neq \phi$. Thus $a \in e(A)$, and since $e(A) \subseteq A$, then $x \in A$.

		In both cases, $a \in A$. By universal generalization, $A$ contains all of its limit points. Therefore $A$ is closed.
	\end{proof}
\end{theorem}
\newpage
\item Let $(x_n)$ be a sequence of real numbers. We say that $(x_n)$ is \emph{mostly-Cauchy} if
\[ |x_n - x_{n+1}| \xrightarrow{n \to \infty} 0.\]
Note that mostly-Cauchy sequences need not converge. For example, $x_n=\log n$ is mostly-Cauchy and divergent.

Now, suppose $(x_n)$ is a mostly-Cauchy sequence that additionally has the property that for all $n \in \mathbb{N}$:
\[ x_{2n} \leq x_{2n+2} \qquad \text{and} \qquad x_{2n+1} \leq x_{2n-1}.\]
Prove that the following results hold for such a sequence:
\begin{enumerate}[(a)]
	\item (5 points) The subsequence $x_{2n}$ is bounded above.
	\item (5 points) The subsequence $x_{2n+1}$ is bounded below.
	\item (10 points) The sequence $(x_n)$ converges.
\end{enumerate}
%-------------- Problem 6 -------------------
\begin{theorem} 
	The sequence $(x_n)$ converges.
	\begin{proof}
		Since $(x_n)$ is mostly-Cauchy, then for any $\epsilon > 0$ there exists an $N_1 \in \mathbb{N}$ such that if $n \geq N_1$, 
		then $|x_n - x_{n+1}| < \frac{\epsilon}{2}$. Also, for any $\epsilon > 0$ there exists an $N_2 \in \mathbb{N}$ such that if $n \geq N_2$, 
		then $|x_{n+1} - x_{n+2}| < \frac{\epsilon}{2}$. So let $N = \max\{N_1, N_2\}$. Then
		\begin{align*}
			|x_n - x_{n+2}| &\leq |x_n - x_{n+1}| + |x_{n+1} - x_{n+2}| \\
			&< \frac{\epsilon}{2} + \frac{\epsilon}{2} \\
			&= \epsilon
		\end{align*}
		Thus, $(x_{2n})$ and $(x_{2n+1})$ are both mostly-cauchy. Since the odd elements of $(x_n)$ are non-increasing, the even elements of $(x_n)$ are non-decreasing, 
		the difference between subsequent odd-indexed elements is decreasing, and the difference between subsequent even-indexed elements is decreasing, $(x_n)$ must converge.
	\end{proof}
\end{theorem}
\end{enumerate}
\end{document} 